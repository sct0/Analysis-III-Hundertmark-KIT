\documentclass[11pt, twoside, a4paper]{article}

% Setup
\usepackage[margin=2.4cm, top=3.5cm]{geometry}
\usepackage[utf8]{inputenc}
\usepackage[ngerman]{babel}

% Package imports
\usepackage{amsfonts}
\usepackage{amsmath}
\usepackage{amssymb}
\usepackage{amsthm}
\usepackage{mathtools}
\usepackage{setspace}
\usepackage{float}
\usepackage{enumitem}
\usepackage{hyperref}
\usepackage[pagestyles]{titlesec}
\usepackage{fancyhdr}
\usepackage{colonequals}
\usepackage{caption}
\usepackage{tikz}
\usepackage{marginnote}
\usepackage{etoolbox}
\usepackage{mdframed}
\usepackage{aligned-overset}
\usepackage{esint}
\usepackage{scalerel}

% Font-Encoding
\usepackage[T1]{fontenc}
\usepackage{lmodern}

% TikZ packages
\usetikzlibrary{patterns}

% Theorems
\newtheoremstyle{plain}{}{}{}{}{\bfseries}{.}{ }{}
\theoremstyle{plain}
\newtheorem{blockelement}{Blockelement}[subsection]
\newtheorem{bemerkung}[blockelement]{Bemerkung}
\newtheorem{definition}[blockelement]{Definition}
\newtheorem{lemma}[blockelement]{Lemma}
\newtheorem{satz}[blockelement]{Satz}
\newtheorem{notation}[blockelement]{Notation}
\newtheorem{korollar}[blockelement]{Korollar}
\newtheorem{uebung}[blockelement]{Übung}
\newtheorem{beispiel}[blockelement]{Beispiel}
\newtheorem{folgerung}[blockelement]{Folgerung}
\newtheorem{axiom}[blockelement]{Axiom}
\newtheorem{beobachtung}[blockelement]{Beobachtung}
\newtheorem{konzept}[blockelement]{Konzept}
\newtheorem{visualisierung}[blockelement]{Visualisierung}
\newtheorem{anwendung}[blockelement]{Anwendung}
\newtheorem{skizze}[blockelement]{Skizze}
\newtheorem{genv}[blockelement]{}

% Numbering (equations and conditions)
\numberwithin{equation}{subsection}
\newcommand{\numbereq}[1]{\addtocounter{equation}{1}\tag{\theequation}\label{#1}}
\newcounter{condition}
\renewcommand{\thecondition}{V\arabic{condition}}
\newcommand{\condition}[1]{\hypertarget{#1}{\refstepcounter{condition}(\label{#1}\thecondition})}

% Marginnotes left
\makeatletter
\patchcmd{\@mn@@@marginnote}{\begingroup}{\begingroup\@twosidefalse}{}{\fail}
\reversemarginpar
\makeatother

% Long equations
\allowdisplaybreaks

% \left \right
\newcommand{\set}[1]{\left\{#1\right\}}
\newcommand{\pair}[1]{\left(#1\right)}
\newcommand{\of}[1]{\mathopen{}\mathclose{}\bgroup\left(#1\aftergroup\egroup\right)}
\newcommand{\abs}[1]{\left\lvert#1\right\rvert}
\newcommand{\norm}[1]{\left\lVert#1\right\rVert}
\newcommand{\linterv}[1]{\left[#1\right)}
\newcommand{\rinterv}[1]{\left(#1\right]}
\newcommand{\interv}[1]{\left[#1\right]}
\newcommand{\sprod}[1]{\left<#1\right>}

% Shorten commands
\newcommand{\equivalent}[0]{\Leftrightarrow{}}
\newcommand{\impl}[0]{\Rightarrow{}}
\newcommand{\fromto}{\rightarrow{}}
\newcommand{\definedasequiv}[0]{\ratio\Leftrightarrow{}}
\newcommand{\exclude}[0]{\setminus}
\renewcommand{\emptyset}{\varnothing}
\newcommand{\dif}{\mathop{}\!\mathrm{d}}

\newcommand{\ntoinf}[0]{n\fromto\infty}
\newcommand{\toinf}{\fromto\infty}
\newcommand{\fa}{\;\forall}
\newcommand{\ex}{\;\exists}
\newcommand{\conj}[1]{\overline{#1}}
\newcommand{\comp}[1]{{#1}^{\mathrm{C}}}

\newcommand{\annot}[3][]{\overset{\text{#3}}#1{#2}}
\newcommand{\biglim}[1]{{\displaystyle \lim_{#1}}}
\newcommand{\anf}[1]{\glqq{}#1\grqq}
\newcommand{\OBDA}{o.B.d.A. }
\newcommand{\theoremescape}{\leavevmode}
\newcommand{\aligntoright}[2]{\hfill#1\hspace{#2\textwidth}~}
\newcommand{\horizontalline}[0]{\par\noindent\rule{0.05\textwidth}{0.1pt}\\}
\newcommand{\rgbcolor}[3]{rgb,255:red,#1;green,#2;blue,#3}
\newcommand{\fixedspace}[2]{\makebox[#1][l]{#2}}
\newcommand{\ov}[1]{\overline{#1}}
\newcommand{\un}[1]{\underline{#1}}
\newcommand{\verteq}{\rotatebox{-90}{$~=$}}
\newcommand{\equalto}[2]{\underset{\scriptstyle\overset{\mkern4mu\verteq}}{#1}}
\newcommand{\eqbelow}[1]{\underset{\verteq}{#1}}

\let\Re\relax
\let\Im\relax

% MathOperators
\DeclareMathOperator{\grad}{Grad}
\DeclareMathOperator{\bild}{Bild}
\DeclareMathOperator{\Re}{Re}
\DeclareMathOperator{\Im}{Im}
\DeclareMathOperator{\arcsinh}{arcsinh}
\DeclareMathOperator{\arccosh}{arccosh}
\DeclareMathOperator{\diam}{diam}
\DeclareMathOperator{\fehler}{Fehler}
\DeclareMathOperator{\D}{D\!}
\DeclareMathOperator{\Id}{Id}
\DeclareMathOperator{\op}{op}
\DeclareMathOperator{\rank}{rk}
\DeclareMathOperator{\spann}{Spann}
\DeclareMathOperator{\flaeche}{Fläche}

% Mengenbezeichner
\newcommand{\R}{\mathbb{R}}
\newcommand{\N}{\mathbb{N}}
\newcommand{\C}{\mathbb{C}}
\newcommand{\Z}{\mathbb{Z}}
\newcommand{\Q}{\mathbb{Q}}
\newcommand{\K}{\mathbb{K}}

\newcommand{\mR}{\mathcal{R}}
\newcommand{\mB}{\mathcal{B}}
\newcommand{\mC}{\mathcal{C}}
\newcommand{\mL}{\mathcal{L}}
\newcommand{\mJ}{\mathcal{J}}
\newcommand{\mPC}{\mathcal{PC}}
\newcommand{\mM}{\mathcal{M}}
\newcommand{\mS}{\mathcal{S}}

% Spezielle Symbole
\NewDocumentCommand{\Tau}{e{^_}}{
    \scalerel*{\tau}{X}
    \IfValueT{#1}{^{#1}}
    \IfValueT{#2}{_{\!\!#2}}
}

% Spezielle Commands
\newcommand\imaginarysubsection[1]{
    \refstepcounter{subsection}
    \subsectionmark{#1}
}

% Unfassbar hässlich, aber effektiv für temporäre schnelle Lösungen
\def\:={\coloneqq}
\def\->{\fromto}
\def\=>{\impl}
\def\<={\leq}
\def\>={\geq}

% Envs
\newenvironment{induktionsanfang}{
    \rule{0pt}{3ex}\noindent
    \begin{minipage}[t]{0.11\textwidth}
    {I-Anfang}
    \end{minipage}
    \hfill
    \begin{minipage}[t]{0.89\textwidth}
    }
    {
    \end{minipage}
}
\newenvironment{induktionsvoraussetzung}{
    \rule{0pt}{3ex}\noindent
    \begin{minipage}[t]{0.11\textwidth}
    {I-Vor.}
    \end{minipage}
    \hfill
    \begin{minipage}[t]{0.89\textwidth}
    }
    {
    \end{minipage}
}
\newenvironment{induktionsschritt}{
    \rule{0pt}{3ex}\noindent
    \begin{minipage}[t]{0.11\textwidth}
    {I-Schritt}
    \end{minipage}
    \hfill
    \begin{minipage}[t]{0.89\textwidth}
    }
    {
    \end{minipage}
}

% Section style
\titleformat*{\section}{\LARGE\bfseries}
\titleformat*{\subsection}{\large\bfseries}

% Page styles
\newpagestyle{pagenumberonly}{
    \sethead{}{}{}
    \setfoot[][][\thepage]{\thepage}{}{}
}
\newpagestyle{headfootdefault}{
    \sethead[][][\thesubsection~\textit{\subsectiontitle}]{\thesection~\textit{\sectiontitle}}{}{}
    \setfoot[][][\thepage]{\thepage}{}{}
}
\pagestyle{headfootdefault}

\begin{document}
    \title{\vspace{3cm} Skript zur Vorlesung\\Analysis III\\bei Prof. Dr. Dirk Hundertmark}
    \author{Karlsruher Institut für Technologie}
    \date{Wintersemester 2024/25}
    \maketitle
    \begin{center}
        Dieses Skript ist inoffiziell. Es besteht kein\\Anspruch auf Vollständigkeit oder Korrektheit.
    \end{center}
    \thispagestyle{empty}
    \newpage

    \tableofcontents
    ~\\
    Alle mit [*] markierten Kapitel sind noch nicht Korrektur gelesen und bedürfen eventuell noch Änderungen.

    \newpage
    
    \section{Einleitung: Motivation für Maßtheorie}
    \imaginarysubsection{Einleitung}
    \thispagestyle{pagenumberonly}
    
    Wir wollen in diesem Modul eine Theorie erarbeiten, um Teilmengen des $\R^n$ messen zu können. Außerdem soll diese Zuordnung eines Inhalts bestimmten Anforderungen genügen. Wenn wir zum Beispiel zwei Teilmengen des $\R^2$ $A$ und $B$ betrachten, die disjunkt sind, dann soll nach unserem intuitiven geometrischen Verständnis auch gelten
    \begin{align*}
        \flaeche\of{A\cup B} &= \flaeche\of{A} + \flaeche\of{B}
    \end{align*}

    \noindent Für einfache Teilmengen des $\R^2$ haben wir bereits eine Möglichkeit, deren Flächeninhalt zu messen

    \begin{beispiel}[Meßen eines Rechtecks]
        Im Fall eines Rechteckes $R\subseteq\R^2$ mit den Seitenlängen $a$ und $b$ wissen wir bereits, dass wir einen sinnvollen Flächeninhalt durch
        \begin{align*}
            \flaeche\of{R} &= a\cdot b
        \end{align*}
        berechnen können.
    \end{beispiel}

    \begin{beispiel}[Meßen eines Dreiecks]
        Auch für ein Dreieck $D\subseteq\R^2$ mit Grundfläche $g$ und Höhe $h$ kennen wir die Formel
        \begin{align*}
            \flaeche\of{D} &= \frac{1}{2}gh
        \end{align*}
    \end{beispiel}

    So können wir auch komplexere Formen mittels (abzählbar) unendlich vielen Dreiecken approximieren. Allerdings gibt es auch Fälle, in denen wir dabei auf Schwierigkeiten stoßen (siehe Cantorsches Dikontinuum).\\

    Wir wollen ein Maß finden, also nach unserem Verständnis eine Abbildung $\mu: \mathcal{F} \to \interv{0, \infty}$, wobei $\mathcal{F} \subseteq \mathcal{P}\of{E}$ ein System von Teilmengen von $E\neq\emptyset$ ist. Außerdem soll gelten, dass

    \begin{enumerate}[label=(\roman*)]
        \item $\mu\of{\emptyset} = 0$
        \item Für $A, B\in\mathcal{F}$ mit $A\cap B = \emptyset$ soll gelten $\mu\of{A \cup B} = \mu\of{A} + \mu\of{B}$
        \item ??
    \end{enumerate}

    \newpage

    \section{$\sigma$-Algebren}
    \imaginarysubsection{$\sigma$-Algebra}
    \thispagestyle{pagenumberonly}

    \begin{definition}[$\sigma$-Algebra]
        Sei $E\neq \emptyset$ eine Menge. Eine $\sigma$-Algebra in $E$ ist ein System von Teilmengen $\mathcal{A} \subseteq \mathcal{P}\of{E}$ von $E$ mit folgenden Eingeschaften
        \begin{enumerate}[label=(\roman*)]
            \item $E\in\mathcal{A}$
            \item $A\in\mathcal{A} \impl A^{\mathrm{C}} \coloneqq E \exclude A \in\mathcal{A}$
            \item Für $n\in\N$ und $A_n\in\mathcal{A} \impl \bigcup_{n\in\N} A_n \in\mathcal{A}$. Das heißt $\mathcal{A}$ ist stabil unter (abzählbaren) Vereinigungen
        \end{enumerate}
        Eine Menge $A\in\mathcal{A}$ heißt messbar ($\mathcal{A}$-messbar).
    \end{definition}

    \begin{lemma}[Eigenschaften von $\sigma$-Algebren]
        Sei $\mathcal{A}$ eine $\sigma$-Algebra in $E$. Dann gilt
        \begin{enumerate}[label=(\roman*)]
            \item $\emptyset\in\mathcal{A}$
            \item $A, B\in\mathcal{A} \impl \pair{A\cup B} \in\mathcal{A}$ (das heißt $\mathcal{A}$ ist auch stabil unter endlichen Vereinigungen)
            \item Für eine Folge von Mengen in der Algebra $\pair{A_n}_n \subseteq \mathcal{A}$ gilt $\bigcap_{n=1}^{\infty} A_n \in\mathcal{A}$
            \item $A, B\in\mathcal{A} \impl A \exclude B = A \cap \comp{B} \in \mathcal{A}$
        \end{enumerate}

        \begin{proof}
            \theoremescape
            \begin{enumerate}[label=(\roman*)]
                \item $E\in\mathcal{A} \impl \emptyset = \comp{E} \in\mathcal{A}$
                \item Wir definieren $A_1 \coloneqq A$, $A_2 \coloneqq B$ und $A_i \coloneqq \emptyset$ für $i\geq 3$. Dann gilt
                \begin{align*}
                    A \cup B &= \bigcup_{n\in\N}^{\infty} A_n \in \mathcal{A}
                \end{align*}
                \item $A_n \in \mathcal{A} \impl \comp{\pair{A_n}} \in \mathcal{A} \impl \bigcup_{n=1}^{\infty} \comp{\pair{A_n}} \in \mathcal{A} \impl \comp{\pair{\comp{\pair{\bigcup_{n=1}^{\infty} A_n}}}} \in \mathcal{A} \impl \bigcup_{n=1}^{\infty} A_n\in\mathcal{A}$
                \item $A\exclude B = A \cap \comp{B} = A \cap \comp{B} \cap E \cap E \cap \dots$. Dann gilt nach (iii), dass $A \exclude B \in\mathcal{A}$
            \end{enumerate}
        \end{proof}
    \end{lemma}

    \begin{beispiel}
        Wir betrachten einige Beispiele für $\sigma$-Algebren
        \begin{enumerate}[label=(\alph*)]
            \item Für eine Mengen $E$ ist die Potenzmenge selber $\mathcal{P}\of{E}$ nach Definition immer eine $\sigma$-Algebra über $E$.
            \item $\set{\emptyset, E}$ ist die kleinste $\sigma$-Algebra in $E$.
            \item Für $A \subseteq E$ gilt $\mathcal{A} \coloneqq \set{\emptyset, A, \comp{A}, E}$ ist die kleinste $\sigma$-Algebra, die $A$ enthält.
            \item Sei $E$ überabzählbar. Dann ist $\mathcal{A} \coloneqq \set{A\subseteq E: A \text{ oder }\comp{A}\text{ ist abzählbar}}$ eine $\sigma$-Algebra.
            \item Sei $\mathcal{A}$ eine $\sigma$-Algebra in $E$. Für $F\subseteq E$ beliebig ist $\mathcal{A}_F \coloneqq \set{A \cap F: A\in\mathcal{A}}$ die Spur-$\sigma$-Algebra von $F$.
            \item Seien $E, E'$ nicht-leere Mengen, $f: E\to E'$ eine Funktion und $\mathcal{A}'$ eine $\sigma$-Algebra in $E'$. Dann ist auch
            \begin{align*}
                \mathcal{A} \coloneqq \set{f^{-1}\of{A'}: A'\in\mathcal{A}'}
            \end{align*}
            eine $\sigma$-Algebra.
        \end{enumerate}

        \begin{proof}[Beweis von (d)]
            Wir prüfen die Kriterien
            \begin{enumerate}[label=(\roman*)]
                \item $\comp{E} = \emptyset$ ist abzählbar $\impl E\in\mathcal{A}$
                \item $A\in\mathcal{A} \equivalent A$ oder $\comp{A}$ ist abzählbar $\equivalent \comp{A}$ oder $\comp{\pair{\comp{A}}}$ ist abzählbar $\equivalent \comp{A} \in\mathcal{A}$
                \item Sei $A_n \in\mathcal{A}$ für $n\in\N$. Wir unterscheiden ? Fälle\\
                \textsc{Fall 1}: Alle $A_n$ sind abzählbar. Dann ist auch $\bigcup_{n\in\N} A_n$ abzählbar.\\
                \textsc{Fall 2}: Ein $A_j$ ist überabzählbar. Dann ist aber $\comp{\pair{A_j}}$ abzählbar. Das heißt $\bigcap_{n=1}^{\infty} \comp{A_n} \subseteq \comp{A_j}$ ist abzählbar. Dann ist $\comp{\pair{\bigcup_{n=1}^{\infty} A_n}} = \bigcap_{n=1}^{\infty} \comp{\pair{A_n}}$ abzählbar. Das heißt $\bigcup_{n=1} A_n \in\mathcal{A}$
            \end{enumerate}
        \end{proof}
    \end{beispiel}
    
    \begin{notation}
        Seien $I$ eine beliebige Menge und $A_j$, $j\in I$ eine beliebige Familie von Mengensystemen in $E$. Dann ist
        \begin{align*}
            \bigcap_{j\in I} \mathcal{A}_j \coloneqq \set{A: A \subseteq \mathcal{A}_j \forall j\in I}
        \end{align*}
        der Durchschnitt der $\mathcal{A}_j$.
    \end{notation}

    \begin{satz} % Satz 4
        \label{satz:schnitt-sig-algebra}
        Sei $I$ eine beliebige Menge und $\mathcal{A}_j$ eine $\sigma$-Algebra in $E$. Dann gilt
        \begin{align*}
            \bigcap_{j\in I} \mathcal{A}_j
        \end{align*}
        ist wieder eine $\sigma$-Algebra.
        \begin{proof}
            \theoremescape
            \begin{enumerate}[label=(\roman*)]
                \item $E\in\mathcal{A}_j~\forall j\in I \impl E \subseteq \bigcap_{j\in I} \mathcal{A}_j$
                \item $A \subseteq \bigcap_{j\in I} \mathcal{A}_j \equivalent A \subseteq \mathcal{A}_j~\forall j\in I$. Daraus folgt $\comp{A} \subseteq \mathcal{A}_j~\forall j\in I \impl A \subseteq \bigcap_{j\in I}\mathcal{A_j}$
                \item Sei $A_n \subseteq \bigcap_{j\in I} \mathcal{A}_j$. Dann gilt $A_n \in \mathcal{A}_j~\forall j\in I \impl \bigcup_{n\in\N} A_n \in \mathcal{A_j}~\forall j\in I$
            \end{enumerate}
        \end{proof}
    \end{satz}

    \begin{satz} % Satz 5
        Sei $\zeta \subseteq \mathcal{P}\of{E}$ für $E$ nicht-leer ein Mengensystem von Teilmengen von $E$. Dann existiert eine kleinste $\sigma$-Algebra $\sigma\of{E}$ in $E$, welche $\zeta$ enthält. Das heißt
        \begin{enumerate}[label=(\alph*)]
            \item $\sigma\of{\zeta}$ ist eine $\sigma$-Algebra in $E$
            \item Für eine $\sigma$-Algebra $\mathcal{A}$ in $E$ mit $\zeta\subseteq A$ folgt $\sigma\of{\zeta} \subseteq \mathcal{A}$
        \end{enumerate}
        Wir nennen $\sigma\of{\zeta}$ in diesem Fall die von $\zeta$ erzeugte $\sigma$-Algebra und $\zeta$ den Erzeuger von $\sigma\of{\zeta}$.
        \begin{proof}
            Wir definieren $I \coloneqq \set{\mathcal{A}: \mathcal{A} \text{ ist $\sigma$-Algebra und } \zeta\subseteq\mathcal{A}}$ die Menge aller $\sigma$-Algebren, die $\zeta$ enthalten. Dabei gilt $I$ nicht-leer, da $\mathcal{P}\of{E}\in I$. Damit gilt nach Satz~\ref{satz:schnitt-sig-algebra}, dass
            \begin{align*}
                \sigma\of{\zeta} \coloneqq \bigcap_{\mathcal{A}\in I} \mathcal{A}
            \end{align*}
            eine $\sigma$-Algebra ist. Dabei ist $\zeta \subseteq \sigma\of{\zeta}$ nach Voraussetzung an $I$. Und nach unserem Beweis ist auch Anforderung (b) erfüllt.
        \end{proof}
    \end{satz}

    \begin{beispiel}
        Sei $\zeta \coloneqq \set{A}$. Dann ist $\set{\emptyset, A, \comp{A}, E}$ die von $\zeta$ erzeugte $\sigma$-Algebra.
    \end{beispiel}

    \begin{definition}
        Sei $\mathcal{O}_d$ das System der offenen Mengen im $\R^d$. Dann definieren wir die \textit{Borel-$\sigma$-Algebra}
        \begin{align*}
            \mathcal{B}_d = \mathcal{B}\of{\R^d} \coloneqq \sigma\of{\mathcal{O}_d}
        \end{align*}
    \end{definition}


\end{document}
