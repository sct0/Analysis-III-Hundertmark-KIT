\documentclass[11pt, twoside, a4paper]{article}

% Setup
\usepackage[margin=2.4cm, top=3.5cm]{geometry}
\usepackage[utf8]{inputenc}
\usepackage[ngerman]{babel}

% Package imports
\usepackage{amsfonts}
\usepackage{amsmath}
\usepackage{amssymb}
\usepackage{amsthm}
\usepackage{mathtools}
\usepackage{setspace}
\usepackage{float}
\usepackage{enumitem}
\usepackage{hyperref}
\usepackage[pagestyles]{titlesec}
\usepackage{fancyhdr}
\usepackage{colonequals}
\usepackage{caption}
\usepackage{tikz}
\usepackage{marginnote}
\usepackage{etoolbox}
\usepackage{mdframed}
\usepackage{aligned-overset}
\usepackage{esint}
\usepackage{scalerel}

% Font-Encoding
\usepackage[T1]{fontenc}
\usepackage{lmodern}

% TikZ packages
\usetikzlibrary{patterns}

% Theorems
\newtheoremstyle{plain}{}{}{}{}{\bfseries}{.}{ }{}
\theoremstyle{plain}
\newtheorem{blockelement}{Blockelement}[subsection]
\newtheorem{bemerkung}[blockelement]{Bemerkung}
\newtheorem{definition}[blockelement]{Definition}
\newtheorem{lemma}[blockelement]{Lemma}
\newtheorem{satz}[blockelement]{Satz}
\newtheorem{notation}[blockelement]{Notation}
\newtheorem{korollar}[blockelement]{Korollar}
\newtheorem{uebung}[blockelement]{Übung}
\newtheorem{beispiel}[blockelement]{Beispiel}
\newtheorem{folgerung}[blockelement]{Folgerung}
\newtheorem{axiom}[blockelement]{Axiom}
\newtheorem{beobachtung}[blockelement]{Beobachtung}
\newtheorem{konzept}[blockelement]{Konzept}
\newtheorem{visualisierung}[blockelement]{Visualisierung}
\newtheorem{anwendung}[blockelement]{Anwendung}
\newtheorem{skizze}[blockelement]{Skizze}
\newtheorem{genv}[blockelement]{}

% Numbering (equations and conditions)
\numberwithin{equation}{subsection}
\newcommand{\numbereq}[1]{\addtocounter{equation}{1}\tag{\theequation}\label{#1}}
\newcounter{condition}
\renewcommand{\thecondition}{V\arabic{condition}}
\newcommand{\condition}[1]{\hypertarget{#1}{\refstepcounter{condition}(\label{#1}\thecondition})}

% Marginnotes left
\makeatletter
\patchcmd{\@mn@@@marginnote}{\begingroup}{\begingroup\@twosidefalse}{}{\fail}
\reversemarginpar
\makeatother

% Long equations
\allowdisplaybreaks

% \left \right
\newcommand{\set}[1]{\left\{#1\right\}}
\newcommand{\pair}[1]{\left(#1\right)}
\newcommand{\of}[1]{\mathopen{}\mathclose{}\bgroup\left(#1\aftergroup\egroup\right)}
\newcommand{\abs}[1]{\left\lvert#1\right\rvert}
\newcommand{\norm}[1]{\left\lVert#1\right\rVert}
\newcommand{\linterv}[1]{\left[#1\right)}
\newcommand{\rinterv}[1]{\left(#1\right]}
\newcommand{\interv}[1]{\left[#1\right]}
\newcommand{\sprod}[1]{\left<#1\right>}

% Shorten commands
\newcommand{\equivalent}[0]{\Leftrightarrow{}}
\newcommand{\impl}[0]{\Rightarrow{}}
\newcommand{\definedasequiv}[0]{\ratio\Leftrightarrow{}}
\renewcommand{\emptyset}{\varnothing}
\newcommand{\dif}{\mathop{}\!\mathrm{d}}
\newcommand{\dsty}{\displaystyle}

\newcommand{\toinf}{\to\infty}
\newcommand{\fa}{\;\forall}
\newcommand{\ex}{\;\exists}
\newcommand{\conj}[1]{\overline{#1}}
\newcommand{\comp}[1]{{#1}^{\mathrm{C}}}

\newcommand{\annot}[3][]{\overset{\text{#3}}#1{#2}}
\newcommand{\anf}[1]{\glqq{}#1\grqq}
\newcommand{\OBDA}{o.B.d.A. }
\newcommand{\theoremescape}{\leavevmode}
\newcommand{\aligntoright}[2]{\hfill#1\hspace{#2\textwidth}~}
\newcommand{\horizontalline}[0]{\par\noindent\rule{0.05\textwidth}{0.1pt}\\}
\newcommand{\rgbcolor}[3]{rgb,255:red,#1;green,#2;blue,#3}
\newcommand{\fixedspace}[2]{\makebox[#1][l]{#2}}

\let\Re\relax
\let\Im\relax

% MathOperators
\DeclareMathOperator{\grad}{Grad}
\DeclareMathOperator{\bild}{Bild}
\DeclareMathOperator{\Re}{Re}
\DeclareMathOperator{\Im}{Im}
\DeclareMathOperator{\arcsinh}{arcsinh}
\DeclareMathOperator{\arccosh}{arccosh}
\DeclareMathOperator{\diam}{diam}
\DeclareMathOperator{\fehler}{Fehler}
\DeclareMathOperator{\D}{D\!}
\DeclareMathOperator{\Id}{Id}
\DeclareMathOperator{\op}{op}
\DeclareMathOperator{\rank}{rk}
\DeclareMathOperator{\spann}{Spann}
\DeclareMathOperator{\flaeche}{Fläche}

% Mengenbezeichner
\newcommand{\R}{\mathbb{R}}
\newcommand{\N}{\mathbb{N}}
\newcommand{\C}{\mathbb{C}}
\newcommand{\Z}{\mathbb{Z}}
\newcommand{\Q}{\mathbb{Q}}
\newcommand{\K}{\mathbb{K}}

\newcommand{\mA}{\mathcal{A}}
\newcommand{\mB}{\mathcal{B}}
\newcommand{\mC}{\mathcal{C}}
\newcommand{\mD}{\mathcal{D}}
\newcommand{\mE}{\mathcal{E}}
\newcommand{\mG}{\mathcal{G}}
\newcommand{\mJ}{\mathcal{J}}
\newcommand{\mK}{\mathcal{K}}
\newcommand{\mL}{\mathcal{L}}
\newcommand{\mM}{\mathcal{M}}
\newcommand{\mO}{\mathcal{O}}
\newcommand{\mP}{\mathcal{P}}
\newcommand{\mQ}{\mathcal{Q}}
\newcommand{\mR}{\mathcal{R}}
\newcommand{\mS}{\mathcal{S}}
\newcommand{\mPC}{\mathcal{PC}}

% Spezielle Symbole
\NewDocumentCommand{\Tau}{e{^_}}{
    \scalerel*{\tau}{X}
    \IfValueT{#1}{^{#1}}
    \IfValueT{#2}{_{\!\!#2}}
}

% Spezielle Commands
\newcommand\imaginarysubsection[1]{
    \refstepcounter{subsection}
    \subsectionmark{#1}
}

% Unfassbar hässlich, aber effektiv für temporäre schnelle Lösungen
\def\:={\coloneqq}
\def\->{\to}
\def\=>{\impl}
\def\<={\leq}
\def\>={\geq}

% Envs
\newenvironment{induktionsanfang}{
    \rule{0pt}{3ex}\noindent
    \begin{minipage}[t]{0.11\textwidth}
    {I-Anfang}
    \end{minipage}
    \hfill
    \begin{minipage}[t]{0.89\textwidth}
    }
    {
    \end{minipage}
}
\newenvironment{induktionsvoraussetzung}{
    \rule{0pt}{3ex}\noindent
    \begin{minipage}[t]{0.11\textwidth}
    {I-Vor.}
    \end{minipage}
    \hfill
    \begin{minipage}[t]{0.89\textwidth}
    }
    {
    \end{minipage}
}
\newenvironment{induktionsschritt}{
    \rule{0pt}{3ex}\noindent
    \begin{minipage}[t]{0.11\textwidth}
    {I-Schritt}
    \end{minipage}
    \hfill
    \begin{minipage}[t]{0.89\textwidth}
    }
    {
    \end{minipage}
}

% Section style
\titleformat*{\section}{\LARGE\bfseries}
\titleformat*{\subsection}{\large\bfseries}

% Page styles
\newpagestyle{pagenumberonly}{
    \sethead{}{}{}
    \setfoot[][][\thepage]{\thepage}{}{}
}
\newpagestyle{headfootdefault}{
    \sethead[][][\thesubsection~\textit{\subsectiontitle}]{\thesection~\textit{\sectiontitle}}{}{}
    \setfoot[][][\thepage]{\thepage}{}{}
}
\pagestyle{headfootdefault}

\begin{document}
    \title{\vspace{3cm} Skript zur Vorlesung\\Analysis III\\bei Prof. Dr. Dirk Hundertmark}
    \author{Karlsruher Institut für Technologie}
    \date{Wintersemester 2024/25}
    \maketitle
    \begin{center}
        Dieses Skript ist inoffiziell. Es besteht kein\\Anspruch auf Vollständigkeit oder Korrektheit.
    \end{center}
    \thispagestyle{empty}
    \newpage

    \tableofcontents
    ~\\
    Alle mit [*] markierten Kapitel sind noch nicht Korrektur gelesen und bedürfen eventuell noch Änderungen.

    \newpage

    \section{Einleitung: Motivation für Maßtheorie}
\imaginarysubsection{Einleitung}
\thispagestyle{pagenumberonly}

\marginnote{[21. Okt]}

Wir wollen in diesem Modul eine Theorie erarbeiten, um Teilmengen des $\R^n$ messen (das heißt ihnen einen Inhalt zuordnen) zu können. Außerdem soll diese Zuordnung eines Inhalts bestimmten (intuitiv klaren) Anforderungen genügen. Wenn wir zum Beispiel zwei Teilmengen des $\R^2$ $A$ und $B$, die disjunkt sind und denen wir entsprechende Inhalte zugeordnet haben, betrachten, dann soll nach unserem intuitiven geometrischen Verständnis auch gelten
\begin{align*}
    \flaeche\of{A\cup B} &= \flaeche\of{A} + \flaeche\of{B}
\end{align*}

\noindent Für einfache Teilmengen des $\R^2$ haben wir bereits eine Möglichkeit, deren Flächeninhalt zu messen:

\begin{beispiel}[Messen eines Rechtecks]
    Im Fall eines Rechteckes $R\subseteq\R^2$ mit den Seitenlängen $a$ und $b$ wissen wir bereits, dass wir einen sinnvollen Flächeninhalt durch
    \begin{align*}
        \flaeche\of{R} &= a\cdot b
    \end{align*}
    berechnen können.
\end{beispiel}

\begin{beispiel}[Messen eines Dreiecks]
    Auch für ein Dreieck $D\subseteq\R^2$ mit Grundfläche $g$ und Höhe $h$ kennen wir die Formel
    \begin{align*}
        \flaeche\of{D} &= \frac{1}{2}gh
    \end{align*}
\end{beispiel}

\begin{beispiel}[Parkettierung]
    Wir können auch eine komplexere Form $F \subseteq\R^2$ mittels (abzählbar) unendlich vielen Dreiecken approximieren. Dafür nehmen wir abzählbar viele paarweise disjunkte Dreiecke $(\Delta_n)_n$, sodass $\dsty\bigcup_{j\in\N} \Delta_j = F$. Dann gilt
    \begin{align*}
        \flaeche\of{F} &= \flaeche\of{\bigcup_{j\in\N} \Delta_j} \annot{=}{(\footnotemark)} \sum_{j=1}^{\infty} \flaeche\of{\Delta_j}
    \end{align*}
    \footnotetext{$\sigma$-Additivität}
\end{beispiel}

\begin{bemerkung}
    Wir wollen dementsprechend ein Maß finden, also nach unserem Verständnis eine Abbildung $\mu: \mathcal{F} \to \interv{0, \infty}$, wobei $\mathcal{F} \subseteq \mathcal{P}\of{E} \coloneqq \set{U: U \subseteq E}$ eine Familie von Teilmengen von $E\neq\emptyset$ ist. Außerdem soll gelten, dass

    \begin{enumerate}[label=(\roman*)]
        \item $\mu\of{\emptyset} = 0$
        \item Für $A, B\in\mathcal{F}$ mit $A\cap B = \emptyset$ ist $\mu\of{A \cup B} = \mu\of{A} + \mu\of{B}$
        \item Für eine Folge $A_n \in \mathcal{F}$ mit $A_n \cap A_m = \emptyset$ für $n\neq m$ ist
        \begin{align*}
            \mu\of{\dsty\bigcup_{n\in\N} A_n} = \sum_{j=1}^{\infty} \mu\of{A_n}
        \end{align*}
    \end{enumerate}
    Diese Liste an Eigenschaften führt wie wir später sehen werden zu einer reichhaltigen Theorie
\end{bemerkung}


\newpage
    \section{$\sigma$-Algebren und Maße}

\subsection{$\sigma$-Algebren}
\thispagestyle{pagenumberonly}

\begin{definition}[$\sigma$-Algebra]
    Sei $E\neq \emptyset$ eine Menge. Eine $\sigma$-Algebra in $E$ ist ein System von Teilmengen $\mathcal{A} \subseteq \mathcal{P}\of{E}$ von $E$ mit folgenden Eigenschaften
    \begin{enumerate}[label=($\Sigma_{\arabic*}$)]
        \item $E\in\mathcal{A}$
        \item $A\in\mathcal{A} \impl A^{\mathrm{C}} \coloneqq E \setminus A \in\mathcal{A}$
        \item Für $(A_n)_n\subseteq\mathcal{A}$ gilt $\dsty\bigcup_{n\in\N} A_n \in\mathcal{A}$. Das heißt $\mathcal{A}$ ist stabil unter (abzählbaren) Vereinigungen
    \end{enumerate}
    Eine Menge $A\in\mathcal{A}$ heißt messbar ($\mathcal{A}$-messbar).
\end{definition}

\begin{lemma}[Eigenschaften von $\sigma$-Algebren]
    Sei $\mathcal{A}$ eine $\sigma$-Algebra in $E$. Dann gilt
    \begin{enumerate}[label=(\roman*)]
        \item $\emptyset\in\mathcal{A}$
        \item $A, B\in\mathcal{A} \impl \pair{A\cup B} \in\mathcal{A}$ (das heißt $\mathcal{A}$ ist auch stabil unter endlichen Vereinigungen)
        \item Für $(A_n)_n \subseteq \mathcal{A}$ gilt $\dsty\bigcap_{n\in\N} A_n \in\mathcal{A}$
        \item $A, B\in\mathcal{A} \impl A \setminus B = A \cap \comp{B} \in \mathcal{A}$
    \end{enumerate}

    \begin{proof}
        \theoremescape
        \begin{enumerate}[label=(\roman*)]
            \item $E\in\mathcal{A} \overset{(\Sigma_2)}{\impl} \emptyset = \comp{E} \in\mathcal{A}$
            \item Wir definieren $A_1 \coloneqq A$, $A_2 \coloneqq B$ und $A_i \coloneqq \emptyset$ für $i\geq 3$. Dann gilt gilt ($\Sigma_3$)
            \begin{align*}
                A \cup B &= \bigcup_{n\in\N} A_n \in \mathcal{A}
            \end{align*}
            \item $A_n \in \mathcal{A} \impl \comp{\pair{A_n}} \in \mathcal{A} \impl \dsty\bigcup_{n\in\N} \comp{\pair{A_n}} \in \mathcal{A} \impl \comp{\pair{\comp{\pair{\bigcup_{n\in\N} A_n}}}} \in \mathcal{A} \impl \bigcap_{n\in\N} A_n\in\mathcal{A}$
            \item $A\setminus B = A \cap \comp{B} = A \cap \comp{B} \cap E \cap E \cap \cdots$. Dann gilt nach (iii), dass $A \setminus B \in\mathcal{A}$\qedhere
        \end{enumerate}
    \end{proof}
\end{lemma}

\begin{beispiel}
    \label{beispiel:sigma-algebren}
    Wir betrachten einige Beispiele für $\sigma$-Algebren
    \begin{enumerate}[label=(\alph*)]
        \item Für eine Mengen $E$ ist die Potenzmenge $\mathcal{P}\of{E}$ selber nach Definition immer eine $\sigma$-Algebra über $E$.
        \item $\set{\emptyset, E}$ ist die kleinste $\sigma$-Algebra in $E$.
        \item Für $A \subseteq E$ gilt $\mathcal{A} \coloneqq \set{\emptyset, A, \comp{A}, E}$ ist die kleinste $\sigma$-Algebra, die $A$ enthält.
        \item Sei $E$ überabzählbar. Dann ist $\mathcal{A} \coloneqq \set{A\subseteq E: A \text{ oder }\comp{A}\text{ ist abzählbar}}$ eine $\sigma$-Algebra.
        \item Sei $\mathcal{A}$ eine $\sigma$-Algebra in $E$. Für $F\subseteq E$ beliebig ist $\mathcal{A}_F \coloneqq \set{A \cap F: A\in\mathcal{A}}$ die Spur-$\sigma$-Algebra von $F$.
        \item Seien $E, E'$ nicht-leere Mengen, $f: E\to E'$ eine Funktion und $\mathcal{A}'$ eine $\sigma$-Algebra in $E'$. Dann ist auch
        \begin{align*}
            \mathcal{A} \coloneqq \set{f^{-1}\of{A'}: A'\in\mathcal{A}'}
        \end{align*}
        eine $\sigma$-Algebra.
    \end{enumerate}

    \begin{proof}[Beweis von (d)]
        Wir prüfen die Kriterien
        \begin{enumerate}[label=($\Sigma_{\arabic*}$)]
            \item $\comp{E} = \emptyset$ ist abzählbar $\impl E\in\mathcal{A}$
            \item $A\in\mathcal{A} \equivalent A$ oder $\comp{A}$ ist abzählbar $\equivalent \comp{A}$ oder $\comp{\pair{\comp{A}}}$ ist abzählbar $\equivalent \comp{A} \in\mathcal{A}$
            \item Sei $A_n \in\mathcal{A}$ für $n\in\N$. Wir unterscheiden 2 Fälle\\
            \textsc{Fall 1}: Alle $A_n$ sind abzählbar. Dann ist auch $\bigcup_{n\in\N} A_n$ abzählbar.\\
            \textsc{Fall 2}: Ein $A_j$ ist überabzählbar. Dann ist aber $\comp{\pair{A_j}}$ abzählbar $\impl\bigcap_{n\in\N} \comp{\pair{A_n}} \subseteq \comp{\pair{A_j}}$ ist abzählbar. Dann ist $\comp{\pair{\bigcup_{n=1}^{\infty} A_n}} = \bigcap_{n\in\N} \comp{\pair{A_n}}$ abzählbar. Das heißt $\bigcup_{n\in\N} A_n \in\mathcal{A}$.\qedhere
        \end{enumerate}
    \end{proof}
\end{beispiel}

\begin{notation}[Durchschnitt]
    Seien $I$ eine beliebige Menge und $(\mA_j)_{j\in I} \subseteq\mathcal{P}\of{E}$ eine beliebige Familie von Mengensystemen in $E$. Dann ist
    \begin{align*}
        \bigcap_{j\in I} \mathcal{A}_j \coloneqq \set{A: A \subseteq \mathcal{A}_j~\forall j\in I}
    \end{align*}
    der Durchschnitt der $\mathcal{A}_j$.
\end{notation}

\begin{satz} % Satz 4
    \label{satz:schnitt-sig-algebra}
    Sei $I$ eine beliebige Menge und $(\mA_j)_{j\in I}$ eine Familie von $\sigma$-Algebren in $E$. Dann gilt
    \begin{align*}
        \bigcap_{j\in I} \mathcal{A}_j
    \end{align*}
    ist wieder eine $\sigma$-Algebra.
    \begin{proof}
        \theoremescape
        \begin{enumerate}[label=($\Sigma_{\arabic*}$)]
            \item $E\in\mathcal{A}_j~\forall j\in I \impl E \subseteq \bigcap_{j\in I} \mathcal{A}_j$
            \item $A \in \bigcap_{j\in I} \mathcal{A}_j \equivalent A \in \mathcal{A}_j~\forall j\in I$. Daraus folgt $\comp{A} \in \mathcal{A}_j~\forall j\in I \impl \comp{A} \in \bigcap_{j\in I}\mathcal{A}_j$
            \item Sei $A_n \in \bigcap_{j\in I} \mathcal{A}_j$. Dann gilt $A_n \in \mathcal{A}_j~\forall j\in I \impl \bigcup_{n\in\N} A_n \in \mathcal{A}_j~\forall j\in I$\qedhere
        \end{enumerate}
    \end{proof}
\end{satz}

\begin{satz} % Satz 5
    Sei $\zeta \subseteq \mathcal{P}\of{E}$ für $E$ nicht-leer ein System von Teilmengen von $E$. Dann existiert eine kleinste $\sigma$-Algebra $\sigma\of{E}$ in $E$, welche $\zeta$ enthält. Das heißt
    \begin{enumerate}[label=(\alph*)]
        \item $\sigma\of{\zeta}$ ist eine $\sigma$-Algebra in $E$ und
        \item Für eine $\sigma$-Algebra $\mathcal{A}$ in $E$ mit $\zeta\subseteq A$ folgt $\sigma\of{\zeta} \subseteq \mathcal{A}$
    \end{enumerate}
    Wir nennen $\sigma\of{\zeta}$ in diesem Fall die von $\zeta$ erzeugte $\sigma$-Algebra und $\zeta$ den Erzeuger von $\sigma\of{\zeta}$.
    \begin{proof}
        Wir definieren $I \coloneqq \set{\mathcal{A}: \mathcal{A} \text{ ist $\sigma$-Algebra und } \zeta\subseteq\mathcal{A}}$ die Menge aller $\sigma$-Algebren, die $\zeta$ enthalten. Dabei gilt $I$ nicht-leer, da $\mathcal{P}\of{E}\in I$. Damit gilt nach Satz~\ref{satz:schnitt-sig-algebra}, dass
        \begin{align*}
            \sigma\of{\zeta} \coloneqq \bigcap_{\mathcal{A}\in I} \mathcal{A}
        \end{align*}
        eine $\sigma$-Algebra ist. Dabei ist $\zeta \subseteq \sigma\of{\zeta}$ nach Forderung an $I$. Und nach unserer Konstruktion ist auch Anforderung (b) erfüllt.
    \end{proof}
\end{satz}

\begin{beispiel}
    Sei $\zeta \coloneqq \set{A}$. Dann ist $\set{\emptyset, A, \comp{A}, E}$ die von $\zeta$ erzeugte $\sigma$-Algebra.
\end{beispiel}

\begin{definition}
    Sei $\mathcal{O}_d$ das System der offenen Mengen im $\R^d$. Dann definieren wir die \textit{Borel-$\sigma$-Algebra}
    \begin{align*}
        \mathcal{B}_d = \mathcal{B}\of{\R^d} \coloneqq \sigma\of{\mathcal{O}_d}
    \end{align*}
\end{definition}

\subsection{Maße und Prämaße}

\begin{mdframed}
    \centering
    Sei in diesem Teilkapitel stets $X$ eine Menge.
\end{mdframed}

\begin{notation}[Disjunkte Vereinigung]
    \marginnote{[25. Okt]}
    Seien $A,B$ Mengen mit $A\cap B = \emptyset$. Dann schreiben wir $A \sqcup B \coloneqq A \cup B$ als disjunkte Vereinigung von $A$ und $B$.
\end{notation}

\begin{definition}[Maß]
    Ein (positives) Maß $\mu$ auf $X$ ist eine Funktion $\mu: \mA \to\interv{0,\infty}$ mit
    \begin{enumerate}[label=($\text{M}_{\arabic*}$)]
        \setcounter{enumi}{-1}
        \item $\mA$ ist eine $\sigma$-Algebra.
        \item $\mu\of{\emptyset} = 0$
        \item Sei $(A_n)_{n\in\N}\subseteq\mA$ eine Folge paarweise disjunkter Mengen. Dann folgt
        \begin{align*}
            \mu\of{\bigsqcup_{n\in\N} A_n} &= \sum_{n\in\N}^{} \mu\of{A_n}
        \end{align*}
    \end{enumerate}
\end{definition}

\begin{definition}[Prämaß]
    Ist $\mA\subseteq\mathcal{P}\of{X}$ nicht unbedingt eine $\sigma$-Algebra und $\mu: \mA\to\interv{0,\infty}$ eine Funktion, so heißt $\mu$ Prämaß, falls
    \begin{enumerate}[label=($\text{PM}_{\arabic*}$)]
        \item $\mu\of{\emptyset} = 0$ (das setzt also auch voraus, dass $\emptyset\in\mA$)
        \item Sind $(A_n)_{n}\subseteq\mA$ paarweise disjunkt und $\pair{\bigsqcup_{n\in\N} A_n}\in\mA$, dann folgt
        \begin{align*}
            \mu\of{\bigsqcup_{n\in\N} A_n} &= \sum_{n\in\N}^{} \mu\of{A_n}
        \end{align*}
    \end{enumerate}
\end{definition}

\begin{definition}[Wachsende und fallende Teilmengenfolgen]
    Sei $(A_n)_n$ eine Folge von Teilmengen von $X$. Dann nennen wir $(A_n)_n$
    \begin{enumerate}[label=-]
        \item wachsend, falls $A_n \subseteq A_{n+1}~\forall n\in\N$
        \item fallend, falls $A_{n+1} \subseteq A_{n}~\forall n\in\N$
    \end{enumerate}
\end{definition}

\begin{notation}
    \theoremescape
    \begin{enumerate}
        \item Für eine wachsende Teilmengenfolge $(A_n)_n$ schreiben wir $A_n \nearrow A$, falls $\bigcup_{n=1}^{\infty} A_n = A$.
        \item Für eine fallende Teilmengenfolge $(A_n)_n$ schreiben wir $A_n \searrow A$, falls $\bigcap_{n=1}^{\infty} A_n = A$.
    \end{enumerate}
\end{notation}

\begin{definition}[Messraum und Maßraum]
    Sei $X$ eine Menge, $\mA$ eine $\sigma$-Algebra und $\mu: \mA\to\interv{0,\infty}$ ein Maß.
    \begin{enumerate}
        \item Wir nennen das Paar $\pair{X, \mA}$ einen Messraum.
        \item Wir nennen das Tripel $\pair{X, \mA, \mu}$ einen Maßraum.
        \item Wir nennen $\mu$ endlich und $\pair{X, \mA, \mu}$ einen endlichen Maßraum, falls $\mu\of{X} < \infty$.
        \item Wir nennen $\mu$ Wahrscheinlichkeitsmaß (W-Maß) und $\pair{X, \mA, \mu}$ einen Wahrscheinlichkeitsraum (W-Raum), falls $\mu\of{X} = 1$.
        \item Wir nennen $\mu$ $\sigma$-endlich, falls es eine Folge $(A_n)_n \subseteq \mA$ gibt mit $A_n \nearrow X$ und\\$\mu\of{A_n} < \infty~\forall n\in\N$. In diesem Fall heißt $(A_n)_n$ eine ausschöpfende Folge.
    \end{enumerate}
\end{definition}

\begin{satz}[Eigenschaften von Maßen] % Satz 3
    \label{satz:eigenschaften-mass}
    Seien $\pair{X, \mA, \mu}$ ein Maßraum sowie $A, B, A_n, B_n \in\mA$. Dann gilt
    \begin{enumerate}[label=(\roman*)]
        \item $A \cap B = \emptyset \impl \mu\of{A \sqcup B} = \mu\of{A} + \mu\of{B}$\hfill (Additivität)
        \item $A \subseteq B \impl \mu\of{A} \leq \mu\of{B}$\hfill (Monotonie)
        \item $A \subseteq B$ und $\mu\of{A} < \infty$ $\impl \mu\of{B\setminus A} = \mu\of{B} - \mu\of{A}$
        \item $\mu\of{A\cup B} + \mu\of{A \cap B} = \mu\of{A} + \mu\of{B}$\hfill (Starke Additivität)
        \item $\mu\of{A \cup B} \leq \mu\of{A} + \mu\of{B}$\hfill (Subadditivität)
        \item $(A_n)_n\nearrow A \impl \displaystyle\mu\of{A} = \sup_{n\in\N} \mu\of{A_n} = \lim_{n\toinf} \mu\of{A_n}$\hfill (Stetigkeit von unten)
        \item $(B_n)_n\searrow B$ und $\mu\of{B_1} < \infty$ $\impl \displaystyle\mu\of{B} = \inf_{n\in\N} \mu\of{B_n} = \lim_{n\toinf} \mu\of{B_n}$\hfill (Stetigkeit von oben)
        \item $\displaystyle\mu\of{\bigcup_{n\in\N} A_n} \leq \sum_{n\in\N}^{} \mu\of{A_n}$\hfill ($\sigma$-Subadditivität)
    \end{enumerate}

    \begin{proof}
        \theoremescape
        \begin{enumerate}[label=(\roman*)]
            \item Sei $A_1 \coloneqq A, A_2 \coloneqq B$ und $A_n \coloneqq \emptyset$ für $n\geq 3$. Dann gilt
            \begin{align*}
                A \sqcup B &= \bigsqcup_{n\in\N} A_n\\
                \impl \mu\of{A\sqcup B} &= \sum_{n=1}^{\infty} \mu\of{A_n} = \mu\of{A_1} + \mu\of{A_2} = \mu\of{A} + \mu\of{B}
            \end{align*}
            \item Sei $A \subseteq B$, dann folgt $B = A \sqcup \pair{B\setminus A}$. Mit (i) folgt
            \begin{align*}
                \mu\of{B} &= \mu\of{A \sqcup \pair{B\setminus A}} = \mu\of{A} + \mu\of{B\setminus A} \geq \mu\of{A}
            \end{align*}
            \item $\mu\of{B} = \mu\of{A} + \mu\of{B\setminus A}$. Dann folgt $\mu\of{B\setminus A} = \mu\of{B} - \mu\of{A}$, falls $\mu\of{A} < \infty$.
            \item Es gilt $A \cup B = A \sqcup \pair{B\setminus\pair{A \cap B}}$. Dann folgt
            \begin{align*}
                \mu\of{A \cup B} + \mu\of{A\cap B} &= \mu\of{A\sqcup\pair{B\setminus\pair{A\cap B}}} + \mu\of{A \cap B}\\
                &= \mu\of{A} + \mu\of{B\setminus \pair{A\cap B}} + \mu\of{A \cap B}\\
                &= \mu\of{A} + \mu\of{B} - \mu\of{A \cap B} + \mu\of{A \cap B}\\
                &= \mu\of{A} + \mu\of{B}
            \end{align*}
            \item Aus (iv) folgt $\mu\of{A} + \mu\of{B} = \mu\of{A\cup B} + \mu\of{A \cap B} \geq \mu\of{A \cup B}$
            \item Sei $(A_n)_n$ wachsend. Wir definieren eine neue Folge von Mengen $(F_n)_n$ mit $F_1 \coloneqq A_1$ und $F_n \coloneqq A_n \setminus A_{n-1}$ für $n\geq 2$. Dann sind $F_j$ paarweise disjunkt und es gilt
            \begin{align*}
                \bigcup_{j=1}^{n} A_j &= \bigsqcup_{j=1}^{n} F_j\\
                \impl \mu\of{\bigcup_{n\in\N} A_n} &= \mu\of{\bigsqcup_{j\in\N} F_j} = \sum_{j=1}^{\infty} \mu\of{F_j}\\
                &= \lim_{n\toinf} \sum_{j=1}^{n} \mu\of{F_j} = \lim_{n\toinf} \mu\of{\bigsqcup_{j=1}^{n} F_j}\\
                &= \lim_{n\toinf} \mu\of{A_n} = \sup_{n\in\N} \mu\of{A_n}
            \end{align*}
            \item Sei $(B_n)_n \searrow B$ mit $\mu\of{B_1} < \infty$. Wir definieren $A_n \coloneqq B_1 \setminus B_n \nearrow B_1 \setminus B$ wachsend. Dann gilt nach (vi)
            \begin{align*}
                \mu\of{B_1 \setminus B} &= \lim_{n\toinf} \mu\of{B_1 \setminus B_n}\\
                \impl\mu\of{B_1} - \mu\of{B} &= \lim_{n\toinf} \pair{\mu\of{B_1} - \mu\of{B_n}} = \mu\of{B_1} - \lim_{n\toinf} \mu\of{B_n}\\
                \impl \mu\of{B} &= \lim_{n\toinf} \mu\of{B_n} = \inf_{n\in\N} \mu\of{B_n}
            \end{align*}
            \item Sei $(A_n)_n \subseteq \mA$. Dann ist $A \coloneqq \bigsqcup_{n\in\N} A_n$. Wir definieren $\hat{A}_k \coloneqq \bigcup_{j=1}^{k} A_j$ wachsend. Dann gilt
            \begin{align*}
                \bigcup_{k=1}^{\infty} \hat{A}_k &= \bigcup_{k=1}^{\infty} \bigcup_{n=1}^{k} A_n = \bigcup_{n\in\N} A_n
                \intertext{Nach (v) gilt}
                \mu\of{A} &= \mu\of{\bigcup_{k=1}^{\infty} \hat{A}_k} = \lim_{k\toinf} \mu\of{\hat{A}_k} = \lim_{k\toinf} \mu\of{\bigcup_{j=1}^{k} A_j}\\
                &\leq \lim_{k\toinf} \sum_{j=1}^{k} \mu\of{A_j} = \sum_{n=1}^{\infty} \mu\of{A_n}\qedhere
            \end{align*}
        \end{enumerate}
    \end{proof}
\end{satz}

\begin{bemerkung}
    \theoremescape
    \begin{enumerate}
        \item Wir schreiben statt \anf{paarweise disjunkt} auch kürzer \anf{disjunkt}
        \item Satz~\ref{satz:eigenschaften-mass} überträgt sich auch auf Prämaße, sofern $\mA$ stabil bezüglich Durchschnitt, Vereinigung und Mengendifferenz ist (für (i)-(iv)) und sofern $\mA$ stabil bezüglich abzählbaren Schnitten und Vereinigungen ist (für die verbleibenden Eigenschaften)
    \end{enumerate}
\end{bemerkung}

\begin{beispiel}[Dirac-Maß]
    Sei $X$ eine Menge, $\mA$ eine $\sigma$-Algebra in $X$ und $x_0\in X$. Wir definieren
    \begin{align*}
        \delta_{x_0}\of{A} \coloneqq \begin{cases}
                                         0 &x_0\not\in A\\
                                         1 &x_0\in A
        \end{cases}
    \end{align*}
    Dann ist $\delta_{x_0}$ ein Maß in $X$ und wird als \textit{Dirac}-Maß bezeichnet.
\end{beispiel}

\begin{beispiel}
    Sei $\mA\coloneqq\set{A\subseteq\R: A\text{ ist abzählbar oder }\comp{A}\text{ ist abzählbar}}$. Dann ist $\mA$ nach Beispiel~\ref{beispiel:sigma-algebren} (d) eine $\sigma$-Algebra in $\R$. Wir definieren ein Maß auf $\mA$ mit
    \begin{align*}
        \mu\of{A} &\coloneqq \begin{cases}
                                 0 &A\text{ ist abzählbar}\\
                                 1 &A\text{ ist nicht abzählbar}
        \end{cases}
    \end{align*}
\end{beispiel}

\begin{beispiel}[Zählmaß]
    Sei $\pair{X, \mA}$ ein Messraum. Dann definieren wir das Zählmaß
    \begin{align*}
        \abs{A} &\coloneqq \begin{cases}
                               \#A &\text{ falls $A$ endlich}\\
                               \infty &\text{ falls $A$ unendlich}
        \end{cases}
    \end{align*}
    wobei $\#A$ die Anzahl an Elemente in $A$ angibt.
\end{beispiel}

\begin{beispiel}[Diskretes W-Maß]
    Sei $\Omega = \set{\omega_1, \omega_2, \ldots}$ eine abzählbare Menge, $\mA = \mathcal{P}\of{\Omega}$ und $(p_n)_{n\in\N} \subseteq\interv{0,1}$ mit $ \sum_{n\in\N}^{} p_n = 1$. Dann ist
    \begin{align*}
        \mathbb{P}\of{A} &\coloneqq \sum_{n\in\N:~\omega_n\in A}^{} p_n = \sum_{n\in\N}^{} p_n\delta_{\omega_n}\of{A}
    \end{align*}
    ein sogenanntes diskretes W-Maß. Der Raum $\pair{\Omega, \mA, \mathbb{P}}$ heißt diskreter W-Raum.
\end{beispiel}

\begin{bemerkung}[Ring und Algebra]
    \marginnote{[28. Okt]}
    Ein Mengensystem $R \subseteq\mathcal{P}\of{X}$ heißt Ring, wenn folgende Eigenschaften erfüllt sind
    \begin{enumerate}[label=($\text{R}_{\arabic*}$)]
        \item $\emptyset\in R$
        \item $A, B\subseteq R \impl \pair{A \setminus B} \in R$
        \item $A, B\subseteq R \impl \pair{A \cup B} \in R$
    \end{enumerate}
    Ist ferner $X\in R$, dann heißt $R$ Algebra.
\end{bemerkung}

\begin{bemerkung}[Eigenschaften von Mengenringen]
    Es sei $R$ ein Mengenring. Dann gilt
    \begin{enumerate}
        \item Nach der Mengengleichheit $A \cap B = A \setminus \pair{A \setminus B}$ enthält $R$ auch Schnitte.
        \item Wir definieren die symmetrische Mengendifferenz $\Delta: R \times R \to R,~\pair{A, B} \mapsto \pair{A\setminus B} \cup \pair{B\setminus A}$. Dann definiert $\pair{R, \Delta, \cap}$ einen kommutativen Ring im Sinne der Algebra, wobei $\Delta$ der \anf{Addition} und $\cap$ der \anf{Multiplikation} entspricht.
    \end{enumerate}
\end{bemerkung}

\newpage


    \section{[*] Dynkinsysteme}
    \imaginarysubsection{Dynkinsysteme}
    \thispagestyle{pagenumberonly}
    \begin{definition}[Dynkinsystem]
        Ein Mengensystem $\mD \subseteq\mP\of{X}$ heißt Dynkinsystem, falls
        \begin{enumerate}[label=($\text{D}_{\arabic*}$)]
            \item $X\in\mD$
            \item $D\in\mD \impl \comp{D}\in\mD$
            \item Für eine paarweise disjunkte Mengenfolge $(D_n)_n \subseteq \mD \impl \dsty\bigsqcup_{n\in\N} D_n \in\mD$
        \end{enumerate}
    \end{definition}

    \begin{beispiel}
        \theoremescape
        \begin{enumerate}
            \item Jede $\sigma$-Algebra ist ein Dynkinsystem.
            \item Sei $X$ eine $2n$-elementige Menge. Dann ist $\mD \coloneqq \set{A \subseteq X: A\text{ hat eine gerade Anzahl an Elementen}}$ ein Dynkinsystem, aber keine $\sigma$-Algebra.
        \end{enumerate}
    \end{beispiel}

    \begin{lemma}
        Sei $I$ eine beliebige Indexmenge und $(\mD_j)_{j\in I}$ eine Familie von Dynkinsystemen in $X$, dann ist $\displaystyle\bigcap_{j\in I} \mD_j$ wieder ein Dynkinsystem.
        \begin{proof}
        (Übung)
        \end{proof}
    \end{lemma}

    \begin{satz} % Satz 4
        Sei $\mG \subseteq\mP\of{X}$. Dann existiert das kleinste Dynkinsystem $\delta\of{\mG}$, welches $\mG$ enthält. Wir nennen $\delta\of{\mG}$ das von $\mG$ erzeugte Dynkinsystem.
        \begin{proof}
            $\mP\of{X}$ ist ein Dynkinsystem. Wir definieren also
            \begin{align*}
                I &= \set{\mD\subseteq\mP\of{X}: \mD\text{ ist ein Dynkinsystem und }\mG\subseteq\mD} \neq \emptyset
            \end{align*}
            Anschließend setzen wir analog zum Schnitt über $\sigma$-Algebren
            \begin{align*}
                \delta\of{\mG} &\coloneqq \bigcap_{\mD\in I} \mD\qedhere
            \end{align*}
        \end{proof}
    \end{satz}

    \begin{definition}
        Sei $\mD\subseteq\mP\of{X}$. Wir nennen $\mD$ $\cap$-stabil, falls $A, B \in\mD \impl \pair{A \cap B} \in\mD$. Analog dazu nennen wir $\mD$ $\cup$-stabil, falls $A, B \in\mD \impl \pair{A \cup B} \in\mD$.
    \end{definition}

    Frage: Wann ist ein Dynkinsystem eine $\sigma$-Algebra?

    \begin{lemma}
        \label{lemma:dynkin-sigma-equiv}
        Sei $\mD$ ein Dynkinsystem. Dann gilt $\mD$ ist genau dann eine $\sigma$-Algebra, wenn $A, B \in \mD \impl \pair{A \cap B} \in\mD$.

        \begin{proof}
            \theoremescape
            \anf{$\impl$} Sei $\mD$ eine $\sigma$-Algebra. Dann ist $\mD$ ein Dynkinsystem. Seien $A, B\in\mD$. Dann folgt $\comp{A}, \comp{B} \in \mD \impl A \cap B = \comp{\pair{\comp{A} \cup \comp{B}}} \in \mD$.\\[0.5\baselineskip]
            \anf{$\Leftarrow$} Zu zeigen ist Eigenschaft ($\Sigma_3$). Sei $(D_n)_n \subseteq\mD$ eine Mengenfolge. Wir definieren $D_0' \coloneqq \emptyset$ und $D_n' \coloneqq D_1 \cup D_2 \cup \dots \cup D_n$. Dann ist $(D'_n)_n$ eine aufsteigende Folge und es gilt
            \begin{align*}
                \bigcup_{n\in\N} D_n = \bigcup_{n\in\N} D_n' &= \bigsqcup_{n\in\N} \pair{D_n' \setminus D_{n-1}'}
            \end{align*}
            Außerdem ist
            \begin{align*}
                \bigsqcup_{n\in\N} \pair{D_n' \setminus D_{n-1}'} \in &\mD\text{ falls } \pair{D_n' \setminus D_{n-1}'} \in \mD~\forall n\in\N
                \intertext{Und es gilt $D_n' \setminus D_{n-1}' = \pair{D_n' \cap \comp{\pair{D_{n-1}'}}}\in \mD$, falls $D_n' \in \mD~\forall n\in\N_0$. Wir haben also unsere Behauptung gezeigt, wenn wir gezeigt haben, dass $\mD$ $\cup$-stabil ist. Es gilt aber}
                A \cup B &= \comp{\pair{\comp{A} \cap \comp{B}}} \in \mD
            \end{align*}
            Damit ist ($\Sigma_3$) gezeigt.
        \end{proof}
    \end{lemma}

    \begin{satz} % Satz 6
        \label{satz:dynkin-cap-stabil}
        Sei $X$ eine beliebige Menge und $\mG\subseteq\mP\of{X}$. Dann folgt aus $\mG$ ist $\cap$-stabil, dass $\delta\of{\mG}$ $\cap$-stabil ist.
        \begin{proof}
            Wir nehmen ein beliebiges $D \in \delta\of{\mG}$ und definieren
            \begin{align*}
                \mD_D \coloneqq \set{Q\in\mP\of{X}: Q \cap D \in \delta\of{\mG}}
            \end{align*}
            Behauptung: $\mD_D$ ist ein Dynkinsystem. Stimmt diese Behauptung, dann können wir folgendermaßen argumentieren: Da $\mG$ $\cap$-stabil ist, gilt
            \begin{align*}
                \forall G, D\in \mG: G \cap D\in \mG \subseteq \delta\of{\mG}\\
                \equivalent \forall D\in\mG: \mG \subseteq\mD_D\\
                \impl \forall D\in\mG: \delta\of{\mG}\subseteq \delta\of{\mD_D} \annot{=}{(Beh.)} \mD_D\\
                \equivalent \forall D\in\mG\fa G\in\delta\of{\mG}: G \cap D \in\delta\of{\mG}
                \intertext{Aus Symmetriegründen gilt dann}
                \forall G\in\delta\of{\mG}\fa D\in \mG: D \cap G = G\cap D \in \delta\of{\mG}\\
                \equivalent \forall G\in \delta\of{\mG}: \mG \subseteq \mD_{G}\\
                \impl \delta\of{\mG} \subseteq \delta\of{\D_G} = \D_g~\forall G\in\delta\of{\mG}\\
                \equivalent \forall D, G\in\delta\of{\mG}: D \cap G\in \delta\of{\mG}
            \end{align*}
            Das heißt $\delta\of{\mG}$ ist $\sigma$-stabil.\\[.5\baselineskip]
            Wir zeigen noch die Behauptung:
            \begin{enumerate}[label=($\text{D}_\arabic*$)]
                \item Da $X \cap D = D \in\delta\of{\mG}$ folgt $X\in \mD_D$.
                \item Sei $Q\in\mD_D$. Dann ist auch $\comp{Q}\in\mD_D$, denn $\comp{Q} \cap D = \pair{\comp{Q} \cup \comp{D}}\cap D = \comp{\pair{Q \cap D}} \cap D = D \setminus\pair{Q\cap D} \in\delta\of{\mG}$.
                \item (Fehlt, siehe handschriftliches Skript)\qedhere
            \end{enumerate}
        \end{proof}
    \end{satz}

    \begin{korollar}
        \label{korollar:dynkin-sigma}
        Sei $X$ eine beliebige Menge und $\mG \subseteq\mathcal{P}\of{X}$. Wenn $\mG$ $\cap$-stabil ist, dann ist $\delta\of{\mG}$ eine $\sigma$-Algebra und es gilt $\sigma\of{\mG} = \delta\of{\mG}$.

        \begin{proof}
            Nach Satz~\ref{satz:dynkin-cap-stabil} ist $\delta\of{\mG}$ $\cap$-stabil und damit nach Lemma~\ref{lemma:dynkin-sigma-equiv} eine $\sigma$-Algebra. Damit gilt dann $\sigma\of{\mG} \subseteq \delta\of{\mG}$, da $\sigma\of{\mG}$ die kleinste $\sigma$-Algebra ist, die $\mG$ enthält. Außerdem ist $\delta\of{\mG}\subseteq \delta\of{\sigma\of{\mG}} = \sigma\of{\mG}$.
        \end{proof}
    \end{korollar}

    \newpage


    \section{[*] Eindeutigkeit von Maßen und erste Eigenschaften des Lebesgue-Maßes}
    \imaginarysubsection{Eindeutigkeit von Maßen}

    \begin{satz}[Eindeutigkeitssatz]
        \marginnote{[04. Nov]}
        \label{satz:eindeutigkeitssatz}
        Sei $(X, \mA)$ ein beliebiger Messraum und $\mA = \sigma\of{\mE}$ für $\mE \subseteq\mP\of{X}$. Ferner seien $\mu, \nu$ Maße auf $\mA$ mit
        \begin{enumerate}[label=(\alph*)]
            \item $\mE$ ist $\cap$-stabil
            \item Es gibt Mengen $G_n \in \mE$ mit $G_n \nearrow X$ ($X = \bigcup_{n\in\N} G_n$) mit $\mu\of{G_n}, \nu\of{G_n} < \infty~\forall n\in\N$
        \end{enumerate}
        Dann gilt: Aus $\mu\of{A} = \nu\of{A}~\forall A\in\mE$ folgt $\mu = \nu$ auf $\mA$. Das heißt unter den obigen Voraussetzungen wird ein Maß eindeutig durch seine Werte auf dem Erzeuger definiert.

        \begin{proof}
            Da $\mE$ $\cap$-stabil ist, folgt nach Korollar~\ref{korollar:dynkin-sigma}, dass $\delta\of{\mE} = \sigma\of{\mE} = \mA$. Wir halten $n\in\N$ fest und betrachten
            \begin{align*}
                \mD_n &\coloneqq \set{A\in \mA: \mu\of{G_n \cap A} = \nu\of{G_n \cap A}}
            \end{align*}
            $\mD_n$ ist ein Dynkinsystem:
            \begin{enumerate}[label=($\text{D}_{\arabic*}$)]
                \item Folgt direkt.
                \item Sei $A\in\mD_n$. Dann ist
                \begin{align*}
                    \mu\of{G_n \cap \comp{A}} &= \mu\of{G_n \setminus A} = \mu\of{G_n \setminus\pair{A \cap G_n}}\\
                    &= \mu\of{G_n} - \mu\of{A \cap G_n}\\
                    &= \nu\of{G_n} - \nu\of{A \cap G_n}\\
                    &= \nu\of{G_n \cap \comp{A}}\\
                    \impl \comp{A} &\in \mD_n
                \end{align*}
                \item Sei $(A_m)_m \subseteq \mD_n$ eine Folge paarweise disjunkter Mengen. Dann gilt
                \begin{align*}
                    \mu\of{\pair{\bigsqcup_{m\in\N} A_m} \cap G_n} &= \mu\of{\bigsqcup_{m\in\N} \pair{A_m \cap G_n}} = \sum_{m\in\N}^{} \mu\of{A_m \cap G_n}\\
                    &= \sum_{m\in\N}^{} \nu\of{A_m \cap G_n} = \nu\of{\pair{\bigsqcup_{m\in\N} A_m} \cap G_n}\\
                    \impl \bigsqcup_{m\in\N} A_m &\in \D_n
                \end{align*}
            \end{enumerate}
            Nach Konstruktion von $\mD_n$ gilt $\mD_n \subseteq\mA$. Andererseits ist $\mE\subseteq\mD_n$. Sei $A \in\mE$ und $A \cap G_n \in \mE$, da $\mE$ $\cap$-stabil. Nach Voraussetzung gilt $\nu\of{A \cap G_n} = \mu\of{A \cap G_n}$, also folgt $A \in\mD_n$.\\
            Da $\mE \subseteq\mD_n \impl \sigma\of{\mE} = \delta\of{\mE} \subseteq \delta\of{\mE} = \mD_n$. Damit gilt $\sigma\of{\mE} \subseteq  \mD_n$. Das heißt $\forall A\in\sigma\of{\mE}$ folgt $\mu\of{A \cap G_n} = \nu\of{A \cap G_n}$.\\
            Für $A\in \sigma\of{\mE}$ definieren wir eine aufsteigende Folge $A_n \coloneqq A \cap G_n \nearrow A$. Da $\mu, \nu$ Maße sind, sind sie von unten stetig. Das heißt
            \begin{align*}
                \mu\of{A} &= \lim_{n\toinf} \mu\of{A_n} = \lim_{n\toinf} \mu\of{A \cap G_n}\\
                &= \lim_{n\toinf} \nu\of{A \cap G_n} = \nu\of{A}\qedhere
            \end{align*}
        \end{proof}
    \end{satz}

    \begin{bemerkung}[Ausschöpfende Folgen]
        Wir nennen $(G_n)_n$ im Sinne von Satz~\ref{satz:eindeutigkeitssatz} eine ausschöpfende Folge. Wir nennen ein Maß $\mu$ auf $\mA$ $\sigma$-endlich, wenn es eine Folge $(G_n)_n \subseteq \mA$ gibt mit $G_n \nearrow X$ und $\mu\of{G_n} < \infty~\forall n\in\N$.
    \end{bemerkung}

    \begin{satz}[Eigenschaften der \textit{Borel}mengen]
        In Definition~\ref{definition:sigma-borel} hatten wir $\mB\of{\R^d} \coloneqq \sigma\of{\mO_d}$, wobei $\mO$ das System offener Teilmengen im $\R^d$ war. Wir definieren nun
        \begin{enumerate}[label=-]
            \item $\mA_d$: System der abgeschlossenen Teilmengen im $\R^d$
            \item $\mK_d$: System der kompakten Teilmengen in $\R^d$
        \end{enumerate}
        Dann gilt $\sigma\of{\mK_d} = \sigma\of{\mA_d} = \sigma\of{\mO_d} = \mB\of{\R^d}$.
        \begin{proof}
            \textsc{Schritt 1}: $\sigma\of{\mA_d} = \sigma\of{\mO_d}$ ist klar, da $\sigma$-Algebren stabil unter Komplementbildung sind.\\
            \textsc{Schritt 2}: Es gilt $\mK_d \subseteq \mA_d \impl \sigma\of{\mK_d} \subseteq \sigma\of{\mA_d}$.\\
            \textsc{Schritt 3}: Für $n\in\N$ definieren wir $K_n \coloneqq \set{\abs{x} < n}$. Sei $A \in \mA_d$, dann ist $A\cap K_n$ kompakt und
            \begin{align*}
                \bigcup_{n\in\N} K_n &= \R^d\\
                A &= \bigcup_{n\in\N} \pair{A \cap K_n} \in \sigma\of{\mK_d}\\
                \impl \mA &\subseteq \sigma\of{\mK_d}\\
                \impl \sigma\of{\mA_d} &\subseteq \sigma\of{\sigma\of{\mK_d}} = \sigma\of{\mK_d}\\
                \impl \sigma\of{\mK_d} &= \sigma\of{\mA_d} = \sigma\of{\mO_d}\qedhere
            \end{align*}
        \end{proof}
    \end{satz}

    Im Folgenden nehmen wir an, dass das Lebesgue-Maß $\lambda^d$ auf $\mB\of{\R^d}$ existiert. Wir werden das später noch beweisen, aber entwickeln das Maß nun nach unserem geometrischen Verständnis unter der Annahme, dass es existiert (das tut es) und untersuchen erste Eigenschaften:

    \begin{beobachtung}
        Wir betrachten den Fall $d=1$ und ein halboffenes Intervall $I\coloneqq \linterv{a,b}$. Dann muss gelten $\lambda^1\of{I} = b-a$. Wir betrachten allgemeine $d$ mit $a,b\in\R^d$ wobei $a\leq b$ (das heißt $a_j \leq b_j$). Dann sei
        \begin{align*}
            \linterv{a,b} &\coloneqq \set{x\in\R^d: a_j \leq x_j \leq b_j~\forall j\in\set{1,\ldots, d}}\\
            \intertext{und wir definieren nach unserem geometrischen Verständnis}
            \lambda^d\of{\linterv{a,b}} &\coloneqq \prod_{j=1}^{d} \pair{b_j - a_j}
        \end{align*}
    \end{beobachtung}

    \begin{definition}
        Es sei $J^d \coloneqq \set{\linterv{a,b}: a,b\in\R^d,~a\leq b}$ das Mengensystem der halboffenen Intervalle im $\R^d$.
    \end{definition}

    \begin{bemerkung}[Translationsinvarianz des Lebesgue-Maß]
        Es sei $c\in\R^d$ und wir definieren eine Translation $T_c\of{x} \coloneqq x+c$ mit inverser Funktion $T_c^{-1}$. Dann gilt für ein halboffenes Intervall $I \coloneqq \linterv{a,b}$
        \begin{align*}
            \lambda^d\of{T_c^{-1}\of{I}} &= \lambda^d\of{\linterv{a-c, b-c}}\\
            &= \prod_{j=1}^{d} \pair{b_j - c_j - \pair{a_j - c_j}}\\
            &= \prod_{j=1}^{d} \pair{b_j - a_j} = \lambda^d\of{I}
        \end{align*}
        Das heißt auf $J^d$ ist $\lambda^d$ invariant unter Translation.
    \end{bemerkung}

    \begin{lemma}
        Sei $B \in\mB^d$ eine Borelmenge und $c\in\R^d$. Dann ist $B + c \coloneqq\set{b+c: b\in B} \in\mB^d$.
        \begin{proof}
            Sei $c\in\R^d$ fest. \textsc{Schritt 1}: Wir wenden das \anf{Wünsch-dir-was}-Vorgehen an und definieren
            \begin{align*}
                \mA \coloneqq \set{A \in B^d: A + c \in B^d}
            \end{align*}
            Dann ist $\mA$ eine $\sigma$-Algebra (Übung).\\
            \textsc{Schritt 2}: $\mO_d$ ist translationsinvariant. Das heißt $\mO_d \in \mA \impl \mB^d = \sigma\of{\mO_d} \subseteq \sigma\of{\mA} = \mA$. Das heißt $\mB^d \subseteq \mA$. Damit sind die Borelmengen translationsinvariant.
        \end{proof}
    \end{lemma}

\end{document}
