\documentclass[11pt, twoside, a4paper]{article}

% Setup
\usepackage[margin=2.4cm, top=3.5cm]{geometry}
\usepackage[utf8]{inputenc}
\usepackage[ngerman]{babel}

% Package imports
\usepackage{amsfonts}
\usepackage{amsmath}
\usepackage{amssymb}
\usepackage{amsthm}
\usepackage{mathtools}
\usepackage{setspace}
\usepackage{float}
\usepackage{enumitem}
\usepackage{hyperref}
\usepackage[pagestyles]{titlesec}
\usepackage{fancyhdr}
\usepackage{colonequals}
\usepackage{caption}
\usepackage{tikz}
\usepackage{marginnote}
\usepackage{etoolbox}
\usepackage{mdframed}
\usepackage{aligned-overset}
\usepackage{esint}
\usepackage{scalerel}

% Font-Encoding
\usepackage[T1]{fontenc}
\usepackage{lmodern}

% TikZ packages
\usetikzlibrary{patterns}

% Theorems
\newtheoremstyle{plain}{}{}{}{}{\bfseries}{.}{ }{}
\theoremstyle{plain}
\newtheorem{blockelement}{Blockelement}[subsection]
\newtheorem{bemerkung}[blockelement]{Bemerkung}
\newtheorem{definition}[blockelement]{Definition}
\newtheorem{lemma}[blockelement]{Lemma}
\newtheorem{satz}[blockelement]{Satz}
\newtheorem{notation}[blockelement]{Notation}
\newtheorem{korollar}[blockelement]{Korollar}
\newtheorem{uebung}[blockelement]{Übung}
\newtheorem{beispiel}[blockelement]{Beispiel}
\newtheorem{folgerung}[blockelement]{Folgerung}
\newtheorem{axiom}[blockelement]{Axiom}
\newtheorem{beobachtung}[blockelement]{Beobachtung}
\newtheorem{konzept}[blockelement]{Konzept}
\newtheorem{visualisierung}[blockelement]{Visualisierung}
\newtheorem{anwendung}[blockelement]{Anwendung}
\newtheorem{skizze}[blockelement]{Skizze}
\newtheorem{genv}[blockelement]{}

% Numbering (equations and conditions)
\numberwithin{equation}{subsection}
\newcommand{\numbereq}[1]{\addtocounter{equation}{1}\tag{\theequation}\label{#1}}
\newcounter{condition}
\renewcommand{\thecondition}{V\arabic{condition}}
\newcommand{\condition}[1]{\hypertarget{#1}{\refstepcounter{condition}(\label{#1}\thecondition})}

% Marginnotes left
\makeatletter
\patchcmd{\@mn@@@marginnote}{\begingroup}{\begingroup\@twosidefalse}{}{\fail}
\reversemarginpar
\makeatother

% Long equations
\allowdisplaybreaks

% \left \right
\newcommand{\set}[1]{\left\{#1\right\}}
\newcommand{\pair}[1]{\left(#1\right)}
\newcommand{\of}[1]{\mathopen{}\mathclose{}\bgroup\left(#1\aftergroup\egroup\right)}
\newcommand{\abs}[1]{\left\lvert#1\right\rvert}
\newcommand{\norm}[1]{\left\lVert#1\right\rVert}
\newcommand{\linterv}[1]{\left[#1\right)}
\newcommand{\rinterv}[1]{\left(#1\right]}
\newcommand{\interv}[1]{\left[#1\right]}
\newcommand{\sprod}[1]{\left<#1\right>}

% Shorten commands
\newcommand{\equivalent}[0]{\Leftrightarrow{}}
\newcommand{\impl}[0]{\Rightarrow{}}
\newcommand{\fromto}{\rightarrow{}}
\newcommand{\definedasequiv}[0]{\ratio\Leftrightarrow{}}
\newcommand{\exclude}[0]{\setminus}
\renewcommand{\emptyset}{\varnothing}
\newcommand{\dif}{\mathop{}\!\mathrm{d}}

\newcommand{\toinf}{\fromto\infty}
\newcommand{\fa}{\;\forall}
\newcommand{\ex}{\;\exists}
\newcommand{\conj}[1]{\overline{#1}}
\newcommand{\comp}[1]{{#1}^{\mathrm{C}}}

\newcommand{\annot}[3][]{\overset{\text{#3}}#1{#2}}
\newcommand{\biglim}[1]{{\displaystyle \lim_{#1}}}
\newcommand{\anf}[1]{\glqq{}#1\grqq}
\newcommand{\OBDA}{o.B.d.A. }
\newcommand{\theoremescape}{\leavevmode}
\newcommand{\aligntoright}[2]{\hfill#1\hspace{#2\textwidth}~}
\newcommand{\horizontalline}[0]{\par\noindent\rule{0.05\textwidth}{0.1pt}\\}
\newcommand{\rgbcolor}[3]{rgb,255:red,#1;green,#2;blue,#3}
\newcommand{\fixedspace}[2]{\makebox[#1][l]{#2}}
\newcommand{\ov}[1]{\overline{#1}}
\newcommand{\un}[1]{\underline{#1}}
\newcommand{\verteq}{\rotatebox{-90}{$~=$}}
\newcommand{\equalto}[2]{\underset{\scriptstyle\overset{\mkern4mu\verteq}}{#1}}
\newcommand{\eqbelow}[1]{\underset{\verteq}{#1}}

\let\Re\relax
\let\Im\relax

% MathOperators
\DeclareMathOperator{\grad}{Grad}
\DeclareMathOperator{\bild}{Bild}
\DeclareMathOperator{\Re}{Re}
\DeclareMathOperator{\Im}{Im}
\DeclareMathOperator{\arcsinh}{arcsinh}
\DeclareMathOperator{\arccosh}{arccosh}
\DeclareMathOperator{\diam}{diam}
\DeclareMathOperator{\fehler}{Fehler}
\DeclareMathOperator{\D}{D\!}
\DeclareMathOperator{\Id}{Id}
\DeclareMathOperator{\op}{op}
\DeclareMathOperator{\rank}{rk}
\DeclareMathOperator{\spann}{Spann}
\DeclareMathOperator{\flaeche}{Fläche}

% Mengenbezeichner
\newcommand{\R}{\mathbb{R}}
\newcommand{\N}{\mathbb{N}}
\newcommand{\C}{\mathbb{C}}
\newcommand{\Z}{\mathbb{Z}}
\newcommand{\Q}{\mathbb{Q}}
\newcommand{\K}{\mathbb{K}}

\newcommand{\mR}{\mathcal{R}}
\newcommand{\mB}{\mathcal{B}}
\newcommand{\mC}{\mathcal{C}}
\newcommand{\mL}{\mathcal{L}}
\newcommand{\mJ}{\mathcal{J}}
\newcommand{\mPC}{\mathcal{PC}}
\newcommand{\mM}{\mathcal{M}}
\newcommand{\mS}{\mathcal{S}}
\newcommand{\mA}{\mathcal{A}}

% Spezielle Symbole
\NewDocumentCommand{\Tau}{e{^_}}{
    \scalerel*{\tau}{X}
    \IfValueT{#1}{^{#1}}
    \IfValueT{#2}{_{\!\!#2}}
}

% Spezielle Commands
\newcommand\imaginarysubsection[1]{
    \refstepcounter{subsection}
    \subsectionmark{#1}
}

% Unfassbar hässlich, aber effektiv für temporäre schnelle Lösungen
\def\:={\coloneqq}
\def\->{\fromto}
\def\=>{\impl}
\def\<={\leq}
\def\>={\geq}

% Envs
\newenvironment{induktionsanfang}{
    \rule{0pt}{3ex}\noindent
    \begin{minipage}[t]{0.11\textwidth}
    {I-Anfang}
    \end{minipage}
    \hfill
    \begin{minipage}[t]{0.89\textwidth}
    }
    {
    \end{minipage}
}
\newenvironment{induktionsvoraussetzung}{
    \rule{0pt}{3ex}\noindent
    \begin{minipage}[t]{0.11\textwidth}
    {I-Vor.}
    \end{minipage}
    \hfill
    \begin{minipage}[t]{0.89\textwidth}
    }
    {
    \end{minipage}
}
\newenvironment{induktionsschritt}{
    \rule{0pt}{3ex}\noindent
    \begin{minipage}[t]{0.11\textwidth}
    {I-Schritt}
    \end{minipage}
    \hfill
    \begin{minipage}[t]{0.89\textwidth}
    }
    {
    \end{minipage}
}

% Section style
\titleformat*{\section}{\LARGE\bfseries}
\titleformat*{\subsection}{\large\bfseries}

% Page styles
\newpagestyle{pagenumberonly}{
    \sethead{}{}{}
    \setfoot[][][\thepage]{\thepage}{}{}
}
\newpagestyle{headfootdefault}{
    \sethead[][][\thesubsection~\textit{\subsectiontitle}]{\thesection~\textit{\sectiontitle}}{}{}
    \setfoot[][][\thepage]{\thepage}{}{}
}
\pagestyle{headfootdefault}

\begin{document}
    \title{\vspace{3cm} Skript zur Vorlesung\\Analysis III\\bei Prof. Dr. Dirk Hundertmark}
    \author{Karlsruher Institut für Technologie}
    \date{Wintersemester 2024/25}
    \maketitle
    \begin{center}
        Dieses Skript ist inoffiziell. Es besteht kein\\Anspruch auf Vollständigkeit oder Korrektheit.
    \end{center}
    \thispagestyle{empty}
    \newpage

    \tableofcontents
    ~\\
    Alle mit [*] markierten Kapitel sind noch nicht Korrektur gelesen und bedürfen eventuell noch Änderungen.

    \newpage


    \section{[*] Einleitung: Motivation für Maßtheorie}
    \imaginarysubsection{Einleitung}
    \thispagestyle{pagenumberonly}

    \marginnote{[21. Okt]}

    Wir wollen in diesem Modul eine Theorie erarbeiten, um Teilmengen des $\R^n$ messen (das heißt ihnen einen Inhalt zuordnen) zu können. Außerdem soll diese Zuordnung eines Inhalts bestimmten (intuitiv klaren) Anforderungen genügen. Wenn wir zum Beispiel zwei Teilmengen des $\R^2$ $A$ und $B$, die disjunkt sind und denen wir entsprechende Inhalte zugeordnet haben, betrachten, dann soll nach unserem intuitiven geometrischen Verständnis auch gelten
    \begin{align*}
        \flaeche\of{A\cup B} &= \flaeche\of{A} + \flaeche\of{B}
    \end{align*}

    \noindent Für einfache Teilmengen des $\R^2$ haben wir bereits eine Möglichkeit, deren Flächeninhalt zu messen:

    \begin{beispiel}[Messen eines Rechtecks]
        Im Fall eines Rechteckes $R\subseteq\R^2$ mit den Seitenlängen $a$ und $b$ wissen wir bereits, dass wir einen sinnvollen Flächeninhalt durch
        \begin{align*}
            \flaeche\of{R} &= a\cdot b
        \end{align*}
        berechnen können.
    \end{beispiel}

    \begin{beispiel}[Messen eines Dreiecks]
        Auch für ein Dreieck $D\subseteq\R^2$ mit Grundfläche $g$ und Höhe $h$ kennen wir die Formel
        \begin{align*}
            \flaeche\of{D} &= \frac{1}{2}gh
        \end{align*}
    \end{beispiel}

    So können wir auch komplexere Formen mittels (abzählbar) unendlich vielen Dreiecken approximieren. Allerdings gibt es auch Fälle, in denen wir dabei auf Schwierigkeiten stoßen (siehe Cantorsches Diskontinuum).\\

    Wir wollen ein Maß finden, also nach unserem Verständnis eine Abbildung $\mu: \mathcal{F} \to \interv{0, \infty}$, wobei $\mathcal{F} \subseteq \mathcal{P}\of{E}$ ein System von Teilmengen von $E\neq\emptyset$ ist. Außerdem soll gelten, dass

    \begin{enumerate}[label=(\roman*)]
        \item $\mu\of{\emptyset} = 0$
        \item Für $A, B\in\mathcal{F}$ mit $A\cap B = \emptyset$ soll gelten $\mu\of{A \cup B} = \mu\of{A} + \mu\of{B}$
        \item ??
    \end{enumerate}

    \newpage


    \section{[*] $\sigma$-Algebren und Maße}

    \subsection{$\sigma$-Algebren}
    \thispagestyle{pagenumberonly}

    \begin{definition}[$\sigma$-Algebra]
        Sei $E\neq \emptyset$ eine Menge. Eine $\sigma$-Algebra in $E$ ist ein System von Teilmengen $\mathcal{A} \subseteq \mathcal{P}\of{E}$ von $E$ mit folgenden Eingeschaften
        \begin{enumerate}[label=(\roman*)]
            \item $E\in\mathcal{A}$
            \item $A\in\mathcal{A} \impl A^{\mathrm{C}} \coloneqq E \exclude A \in\mathcal{A}$
            \item Für $n\in\N$ und $A_n\in\mathcal{A} \impl \bigcup_{n\in\N} A_n \in\mathcal{A}$. Das heißt $\mathcal{A}$ ist stabil unter (abzählbaren) Vereinigungen
        \end{enumerate}
        Eine Menge $A\in\mathcal{A}$ heißt messbar ($\mathcal{A}$-messbar).
    \end{definition}

    \begin{lemma}[Eigenschaften von $\sigma$-Algebren]
        Sei $\mathcal{A}$ eine $\sigma$-Algebra in $E$. Dann gilt
        \begin{enumerate}[label=(\roman*)]
            \item $\emptyset\in\mathcal{A}$
            \item $A, B\in\mathcal{A} \impl \pair{A\cup B} \in\mathcal{A}$ (das heißt $\mathcal{A}$ ist auch stabil unter endlichen Vereinigungen)
            \item Für eine Folge von Mengen in der Algebra $\pair{A_n}_n \subseteq \mathcal{A}$ gilt $\bigcap_{n=1}^{\infty} A_n \in\mathcal{A}$
            \item $A, B\in\mathcal{A} \impl A \exclude B = A \cap \comp{B} \in \mathcal{A}$
        \end{enumerate}

        \begin{proof}
            \theoremescape
            \begin{enumerate}[label=(\roman*)]
                \item $E\in\mathcal{A} \impl \emptyset = \comp{E} \in\mathcal{A}$
                \item Wir definieren $A_1 \coloneqq A$, $A_2 \coloneqq B$ und $A_i \coloneqq \emptyset$ für $i\geq 3$. Dann gilt
                \begin{align*}
                    A \cup B &= \bigcup_{n\in\N}^{\infty} A_n \in \mathcal{A}
                \end{align*}
                \item $A_n \in \mathcal{A} \impl \comp{\pair{A_n}} \in \mathcal{A} \impl \bigcup_{n=1}^{\infty} \comp{\pair{A_n}} \in \mathcal{A} \impl \comp{\pair{\comp{\pair{\bigcup_{n=1}^{\infty} A_n}}}} \in \mathcal{A} \impl \bigcup_{n=1}^{\infty} A_n\in\mathcal{A}$
                \item $A\exclude B = A \cap \comp{B} = A \cap \comp{B} \cap E \cap E \cap \dots$. Dann gilt nach (iii), dass $A \exclude B \in\mathcal{A}$
            \end{enumerate}
        \end{proof}
    \end{lemma}

    \begin{beispiel}
        Wir betrachten einige Beispiele für $\sigma$-Algebren
        \begin{enumerate}[label=(\alph*)]
            \item Für eine Mengen $E$ ist die Potenzmenge selber $\mathcal{P}\of{E}$ nach Definition immer eine $\sigma$-Algebra über $E$.
            \item $\set{\emptyset, E}$ ist die kleinste $\sigma$-Algebra in $E$.
            \item Für $A \subseteq E$ gilt $\mathcal{A} \coloneqq \set{\emptyset, A, \comp{A}, E}$ ist die kleinste $\sigma$-Algebra, die $A$ enthält.
            \item Sei $E$ überabzählbar. Dann ist $\mathcal{A} \coloneqq \set{A\subseteq E: A \text{ oder }\comp{A}\text{ ist abzählbar}}$ eine $\sigma$-Algebra.
            \item Sei $\mathcal{A}$ eine $\sigma$-Algebra in $E$. Für $F\subseteq E$ beliebig ist $\mathcal{A}_F \coloneqq \set{A \cap F: A\in\mathcal{A}}$ die Spur-$\sigma$-Algebra von $F$.
            \item Seien $E, E'$ nicht-leere Mengen, $f: E\to E'$ eine Funktion und $\mathcal{A}'$ eine $\sigma$-Algebra in $E'$. Dann ist auch
            \begin{align*}
                \mathcal{A} \coloneqq \set{f^{-1}\of{A'}: A'\in\mathcal{A}'}
            \end{align*}
            eine $\sigma$-Algebra.
        \end{enumerate}

        \begin{proof}[Beweis von (d)]
            Wir prüfen die Kriterien
            \begin{enumerate}[label=(\roman*)]
                \item $\comp{E} = \emptyset$ ist abzählbar $\impl E\in\mathcal{A}$
                \item $A\in\mathcal{A} \equivalent A$ oder $\comp{A}$ ist abzählbar $\equivalent \comp{A}$ oder $\comp{\pair{\comp{A}}}$ ist abzählbar $\equivalent \comp{A} \in\mathcal{A}$
                \item Sei $A_n \in\mathcal{A}$ für $n\in\N$. Wir unterscheiden ? Fälle\\
                \textsc{Fall 1}: Alle $A_n$ sind abzählbar. Dann ist auch $\bigcup_{n\in\N} A_n$ abzählbar.\\
                \textsc{Fall 2}: Ein $A_j$ ist überabzählbar. Dann ist aber $\comp{\pair{A_j}}$ abzählbar. Das heißt $\bigcap_{n=1}^{\infty} \comp{A_n} \subseteq \comp{A_j}$ ist abzählbar. Dann ist $\comp{\pair{\bigcup_{n=1}^{\infty} A_n}} = \bigcap_{n=1}^{\infty} \comp{\pair{A_n}}$ abzählbar. Das heißt $\bigcup_{n=1} A_n \in\mathcal{A}$
            \end{enumerate}
        \end{proof}
    \end{beispiel}

    \begin{notation}
        Seien $I$ eine beliebige Menge und $A_j$, $j\in I$ eine beliebige Familie von Mengensystemen in $E$. Dann ist
        \begin{align*}
            \bigcap_{j\in I} \mathcal{A}_j \coloneqq \set{A: A \subseteq \mathcal{A}_j \forall j\in I}
        \end{align*}
        der Durchschnitt der $\mathcal{A}_j$.
    \end{notation}

    \begin{satz} % Satz 4
        \label{satz:schnitt-sig-algebra}
        Sei $I$ eine beliebige Menge und $\mathcal{A}_j$ eine $\sigma$-Algebra in $E$. Dann gilt
        \begin{align*}
            \bigcap_{j\in I} \mathcal{A}_j
        \end{align*}
        ist wieder eine $\sigma$-Algebra.
        \begin{proof}
            \theoremescape
            \begin{enumerate}[label=(\roman*)]
                \item $E\in\mathcal{A}_j~\forall j\in I \impl E \subseteq \bigcap_{j\in I} \mathcal{A}_j$
                \item $A \subseteq \bigcap_{j\in I} \mathcal{A}_j \equivalent A \subseteq \mathcal{A}_j~\forall j\in I$. Daraus folgt $\comp{A} \subseteq \mathcal{A}_j~\forall j\in I \impl A \subseteq \bigcap_{j\in I}\mathcal{A_j}$
                \item Sei $A_n \subseteq \bigcap_{j\in I} \mathcal{A}_j$. Dann gilt $A_n \in \mathcal{A}_j~\forall j\in I \impl \bigcup_{n\in\N} A_n \in \mathcal{A_j}~\forall j\in I$
            \end{enumerate}
        \end{proof}
    \end{satz}

    \begin{satz} % Satz 5
        Sei $\zeta \subseteq \mathcal{P}\of{E}$ für $E$ nicht-leer ein Mengensystem von Teilmengen von $E$. Dann existiert eine kleinste $\sigma$-Algebra $\sigma\of{E}$ in $E$, welche $\zeta$ enthält. Das heißt
        \begin{enumerate}[label=(\alph*)]
            \item $\sigma\of{\zeta}$ ist eine $\sigma$-Algebra in $E$
            \item Für eine $\sigma$-Algebra $\mathcal{A}$ in $E$ mit $\zeta\subseteq A$ folgt $\sigma\of{\zeta} \subseteq \mathcal{A}$
        \end{enumerate}
        Wir nennen $\sigma\of{\zeta}$ in diesem Fall die von $\zeta$ erzeugte $\sigma$-Algebra und $\zeta$ den Erzeuger von $\sigma\of{\zeta}$.
        \begin{proof}
            Wir definieren $I \coloneqq \set{\mathcal{A}: \mathcal{A} \text{ ist $\sigma$-Algebra und } \zeta\subseteq\mathcal{A}}$ die Menge aller $\sigma$-Algebren, die $\zeta$ enthalten. Dabei gilt $I$ nicht-leer, da $\mathcal{P}\of{E}\in I$. Damit gilt nach Satz~\ref{satz:schnitt-sig-algebra}, dass
            \begin{align*}
                \sigma\of{\zeta} \coloneqq \bigcap_{\mathcal{A}\in I} \mathcal{A}
            \end{align*}
            eine $\sigma$-Algebra ist. Dabei ist $\zeta \subseteq \sigma\of{\zeta}$ nach Voraussetzung an $I$. Und nach unserem Beweis ist auch Anforderung (b) erfüllt.
        \end{proof}
    \end{satz}

    \begin{beispiel}
        Sei $\zeta \coloneqq \set{A}$. Dann ist $\set{\emptyset, A, \comp{A}, E}$ die von $\zeta$ erzeugte $\sigma$-Algebra.
    \end{beispiel}

    \begin{definition}
        Sei $\mathcal{O}_d$ das System der offenen Mengen im $\R^d$. Dann definieren wir die \textit{Borel-$\sigma$-Algebra}
        \begin{align*}
            \mathcal{B}_d = \mathcal{B}\of{\R^d} \coloneqq \sigma\of{\mathcal{O}_d}
        \end{align*}
    \end{definition}

    \subsection{Maße und Prämaße}

    \begin{mdframed}
        \centering
        Sei in diesem Teilkapitel stets $X$ eine Menge.
    \end{mdframed}

    \begin{definition}[Maß]
        \marginnote{[25. Okt]}
        Ein (positives) Maß $\mu$ auf $X$ ist eine Funktion $\mu: \mA \to\interv{0,\infty}$ mit
        \begin{enumerate}[label=(\roman*)]
            \item $\mA$ ist eine $\sigma$-Algebra.
            \item $\mu\of{\emptyset} = 0$
            \item Sei für $n\in\N$ $(A_n)_n\in\mA$ eine Folge paarweise disjunkter Mengen. Dann folgt
            \begin{align*}
                \mu\of{\bigcup_{n\in\N} A_n} &= \sum_{n\in\N}^{} \mu\of{A_n}
            \end{align*}
        \end{enumerate}
    \end{definition}

    \begin{definition}[Prämaß]
        Ist $\mA\subseteq\mathcal{P}\of{X}$ nicht unbedingt eine $\sigma$-Algebra und $\mu: \mA\to\interv{0,\infty}$ eine Funktion, so heißt $\mu$ Prämaß, falls
        \begin{enumerate}[label=(\roman*)]
            \item $\mu\of{\emptyset} = 0$ (das setzt also auch voraus, dass $\emptyset\in\mA$)
            \item Sind $(A_n)_n\in\mA$ paarweise disjunkt und $\pair{\bigcup_{n\in\N} A_n}\in\mA$, dann folgt
            \begin{align*}
                \mu\of{\dot\bigcup_{n\in\N} A_n} &= \sum_{n\in\N}^{} \mu\of{A_n}
            \end{align*}
        \end{enumerate}
    \end{definition}

    \begin{definition}[Wachsende und fallende Teilmengenfolgen]
        Sei $(A_n)_n$ eine Folge von Teilmengen von $X$. Dann nennen wir $(A_n)_n$
        \begin{enumerate}[label=-]
            \item wachsend, falls $A_n \subseteq A_{n+1}~\forall n\in\N$
            \item fallend, falls $A_{n+1} \subseteq A_{n}~\forall n\in\N$
        \end{enumerate}
    \end{definition}

    \begin{notation}
        \theoremescape
        \begin{enumerate}
            \item Für eine wachsende Teilmengenfolge $(A_n)_n$ schreiben wir $A_n \nearrow A$, falls $\bigcup_{n=1}^{\infty} A_n = A$.
            \item Für eine fallende Teilmengenfolge $(A_n)_n$ schreiben wir $A_n \searrow A$, falls $\bigcap_{n=1}^{\infty} A_n = A$.
        \end{enumerate}
    \end{notation}

    \begin{definition}[Messraum und Maßraum]
        Sei $X$ eine Menge, $\mA$ eine $\sigma$-Algebra und $\mu: \mA\to\interv{0,\infty}$ ein Maß.
        \begin{enumerate}
            \item Wir nennen das Paar $\pair{X, \mA}$ einen Messraum.
            \item Wir nennen das Tripel $\pair{X, \mA, \mu}$ einen Maßraum.
            \item Wir nennen $\mu$ endlich und $\pair{X, \mA, \mu}$ einen endlichen Maßraum, falls $\mu\of{X} < \infty$.
            \item Wir nennen $\mu$ Wahrscheinlichkeitsmaß (W-Maß) und $\pair{X, \mA, \mu}$ einen Wahrscheinlichkeitsraum (W-Raum), falls $\mu\of{X} = 1$.
            \item Wir nennen $\mu$ $\sigma$-endlich, falls es eine Folge $(A_n)_n \subseteq \mA$ gibt mit $A_n \nearrow X$ und $\mu\of{A_n} < \infty~\forall n\in\N$. In diesem Fall heißt $(A_n)_n$ eine ausschöpfende Folge.
        \end{enumerate}
    \end{definition}

    \begin{satz}[Eigenschaften von Maßen] % Satz 3
        \label{satz:eigenschaften-mass}
        Seien $\pair{X, \mA, \mu}$ ein Maßraum sowie $A, B, (A_n)_n, \pair{B_n}_n \in\mA$. Dann gilt
        \begin{enumerate}[label=(\roman*)]
            \item $A \cap B = \emptyset \impl \mu\of{A \cup B} = \mu\of{A} + \mu\of{B}$\hfill (Additivität)
            \item $A \subseteq B \impl \mu\of{A} \leq \mu\of{B}$\hfill (Monotonie)
            \item $A \subseteq B$ und $\mu\of{A} < \infty$ $\impl \mu\of{B\exclude A} = \mu\of{B} - \mu\of{A}$
            \item $\mu\of{A\cup B} + \mu\of{A \cap B} = \mu\of{A} + \mu\of{B}$\hfill (Starke Additivität)
            \item $\mu\of{A \cup B} \leq \mu\of{A} + \mu\of{B}$\hfill (Subadditivität)
            \item $(A_n)_n\nearrow A \impl \displaystyle\mu\of{A} = \sup_{n\in\N} \mu\of{A_n} = \lim_{n\toinf} \mu\of{A_n}$\hfill (Stetigkeit von unten)
            \item $(B_n)_n\searrow B$ und $\mu\of{B_1} < \infty$ $\impl \displaystyle\mu\of{B} = \inf_{n\in\N} \mu\of{B_n} = \lim_{n\toinf} \mu\of{B_n}$\hfill (Stetigkeit von oben)
            \item $\displaystyle\mu\of{\bigcup_{n\in\N} A_n} \leq \sum_{n\in\N}^{} \mu\of{A_n}$\hfill ($\sigma$-Subadditivität)
        \end{enumerate}

        \begin{proof}
            \theoremescape
            \begin{enumerate}[label=(\roman*)]
                \item Sei $A_1 = A, A_2 = B$ und $A_n = \emptyset$ für $n\geq 3$. Dann gilt
                \begin{align*}
                    A \dot\cup B &= \dot\bigcup_{n\in\N} A_n\\
                    \impl \mu\of{A\dot\cup B} &= \sum_{n=1}^{\infty} \mu\of{A_n} = \mu\of{A_1} + \mu\of{A_2} = \mu\of{A} + \mu\of{B}
                \end{align*}
                \item Sei $A \subseteq B$, dann folgt $B = A \dot\cup \pair{B\exclude A}$. Mit (i) folgt
                \begin{align*}
                    \mu\of{B} &= \mu\of{A \dot\cup \pair{B\exclude A}} = \mu\of{A} + \mu\of{B\exclude A} \geq \mu\of{A}
                \end{align*}
                \item $\mu\of{B} = \mu\of{A} + \mu\of{B\exclude A}$. Dann folgt $\mu\of{B\exclude A} = \mu\of{B} - \mu\of{A}$, falls $\mu\of{A} < \infty$.
                \item Es gilt $A \cup B = A \dot\cup \pair{B\exclude\pair{A \cap B}}$. Dann folgt
                \begin{align*}
                    \mu\of{A \cup B} + \mu\of{A\cap B} &= \mu\of{A\dot\cup\pair{B\exclude\pair{A\cap B}}} + \mu\of{A \cap B}\\
                    &= \mu\of{A} + \mu\of{B\exclude \pair{A\cap B}} + \mu\of{A \cap B}\\
                    &= \mu\of{A} + \mu\of{B} - \mu\of{A \cap B} + \mu\of{A \cap B}\\
                    &= \mu\of{A} + \mu\of{B}
                \end{align*}
                \item Aus (iv) folgt $\mu\of{A} + \mu\of{B} = \mu\of{A\cup B} + \mu\of{A \cap B} \geq \mu\of{A \cup B}$
                \item Sei $(A_n)_n$ wachsend. Wir definieren eine neue Folge von Mengen $(F_n)_n$ mit $F_1 \coloneqq A_1$ und $F_n \coloneqq A_n \setminus A_{n-1}$ für $n\geq 2$. Dann sind $F_j$ paarweise disjunkt und es gilt
                \begin{align*}
                    \bigcup_{j=1}^{n} A_j &= \dot\bigcup_{j=1}^{n} F_j\\
                    \impl \mu\of{\bigcup_{n\in\N} A_n} &= \mu\of{\dot\bigcup_{n\in\N} F_j} = \sum_{j=1}^{\infty} \mu\of{F_j}\\
                    &= \lim_{n\toinf} \sum_{j=1}^{n} \mu\of{F_j} = \lim_{n\toinf} \mu\of{\bigcup_{j=1}^{n} F_j}\\
                    &= \lim_{n\toinf} \mu\of{A_n} = \sup_{n\in\N} \mu\of{A_n}
                \end{align*}
                \item Sei $(B_n)_n \searrow B$ mit $\mu\of{B_1} < \infty$. Wir definieren $A_n \coloneqq B_1 \exclude B_n \nearrow B_1 \exclude B$ wachsend. Dann gilt nach (vi)
                \begin{align*}
                    \mu\of{B_1 \exclude B} &= \lim_{n\toinf} \mu\of{B_1 \exclude B_n}\\
                    \mu\of{B_1} - \mu\of{B} &= \lim_{n\toinf} \pair{\mu\of{B_1} - \mu\of{B_n}} = \mu\of{B} - \lim_{n\toinf} \mu\of{B_n}\\
                    \impl \mu\of{B} &= \lim_{n\toinf} \mu\of{B_n} = \inf_{n\in\N} \mu\of{B_n}
                \end{align*}
                \item Sei $(A_n)_n \subseteq \mA$. Dann ist $A = \dot\bigcup_{n\in\N} A_n$. Wir definieren $\hat{A}_k \coloneqq \bigcup_{j=1}^{k} A_j$ wachsend. Dann gilt
                \begin{align*}
                    \bigcup_{k=1}^{\infty} \hat{A}_k &= \bigcup_{k=1}^{\infty} \bigcup_{n=1}^{k} A_n = \bigcup_{n\in\N} A_n
                    \intertext{Nach (v) gilt}
                    \mu\of{A} = \mu\of{\bigcup_{k=1}^{\infty} \hat{A}_k} &= \lim_{k\toinf} \mu\of{\hat{A}_k}\\
                    &= \lim_{k\toinf} \mu\of{\bigcup_{j=1}^{k} A_j} \leq \lim_{k\toinf} \sum_{j=1}^{k} \mu\of{A_j}\\
                    &\leq \lim_{k\toinf} \sum_{j=1}^{k} \mu\of{A_j} = \sum_{n=1}^{\infty} \mu\of{A_n}
                \end{align*}
            \end{enumerate}
        \end{proof}
    \end{satz}

    \begin{bemerkung}
        \theoremescape
        \begin{enumerate}
            \item Wir schreiben statt \anf{paarweise disjunkt} auch kürzer \anf{disjunkt}
            \item Satz~\ref{satz:eigenschaften-mass} überträgt sich auch auf Prämaße, sofern $A$ stabil bezüglich Durchschnitt, Vereinigung und Mengendifferenz ist (für (i)-(iv)) und ? (für die verbleibenden Eigenschaften)
        \end{enumerate}
    \end{bemerkung}

    \begin{beispiel}[Dirac-Maß]
        Sei $X$ eine Menge, $\mA$ eine $\sigma$-Algebra in $X$ und $x_0\in X$. Wir definieren
        \begin{align*}
            \delta_{x_0}\of{A} \coloneqq \begin{cases}
                                             0 &x_0\not\in A\\
                                             1 &x_0\in A
            \end{cases}
        \end{align*}
        Dann ist $\delta_{x_0}$ ein Maß in $X$ und wird als \textit{Dirac}-Maß bezeichnet.
    \end{beispiel}

    \begin{beispiel}[Zählmaß]
        Sei $\pair{X, \mA}$ ein Messraum. Dann definieren wir das Zählmaß
        \begin{align*}
            \abs{A} &\coloneqq \begin{cases}
                                   \#A &\text{ falls $A$ endlich}\\
                                   \infty &\text{ falls $A$ unendlich}
            \end{cases}
        \end{align*}
    \end{beispiel}


\end{document}
